% ************************** Thesis Abstract *****************************
% Use `abstract' as an option in the document class to print only the titlepage and the abstract.
\begin{abstract}
This thesis is consists of two parts. First, we redefined the $60$ years old site percolation 
focusing primarily 
on entropy 
which quantifies the degree of disorder and order parameter
that measures the extent of order. 
Note that being two opposite quantities they can neither be
minimum nor be maximum at the 
same time which is perfectly consistent 
with bond percolation. However, the same is  not 
true for traditional site
percolation as we find that entropy and order parameter are 
both zero at occupation 
probability $p=0$ and the way entropy behaves it violates the 
second law of thermodynamics. To overcome this we redefine the site percolation 
where 
we occupy sites to connect bonds and we measure cluster size by the number of bonds 
connected by occupied sites. 
This resolves the problem without affecting any of the 
existing known results whatsoever.

Second, we investigate percolation by random sequential ballistic deposition (RSBD) on a square lattice 
with interaction range upto second nearest neighbors. In percolaton by random sequential 
deposition process a site is picked at random and we occupy it if the site is empty else
the trial attempt is rejected. In sequential ballistic 
deposition we modify the rejection criteron. That is, instead of rejecting the trial attempt 
we let roll over the occupied particle along one of the 
four directions at random which is then occupied if that site is emtpy else the attempt rejected.  
The critical points $p_c$ and 
all the necessary critical exponents $\alpha$, $\beta$, $\gamma$, $\nu$ etc. are obtained numerically for 
each range of interactions. Like  in its thermal counterpart, we find that the critical exponents 
of RSBD depend on the range of interactions and for a given range of interaction they obey the Rushbrooke inequality. 
Besides, we obtain  the fractal dimension $d_f$ that characterizes 
the spanning cluster at $p_c$. Our results suggest that the RSBD for each range of interaction
belong to a new universality class which is in sharp contrast to earlier results of the only work that exhist 
on RSBD.
\end{abstract}
