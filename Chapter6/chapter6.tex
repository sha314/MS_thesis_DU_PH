
\chapter{Summary and Discussion}

Up until now we knew the exponents for $L0$ or direct interaction. We have investigated for short and long range interactions and we call this $L1$ and $L2$ respectively. Where $L1$ is the case where we can choose the first nearest neighbor and $L2$ is the case where we can chose 2nd nearest neighbor in the direction of first nearest neighbor. We can use new feature of $L1$ only if the feature of $L0$ is unavailable, i.e., the selected site is already occupied. Similarly we can use new feature of $L2$ only if the feature of $L0$ and $L1$ is unavailable. Using this in mind we perform simulation and we obtain the critical exponents which agree with the laws of thermodynamics and the Rushbrooke inequality is satisfied in all cases.\\
The combined exponents are listed below\\
\begin{table}
\centering
\begin{tabular}{|c|c|c|c|c|c|}
	\hline
	Interaction & $p_c$ & $1/\nu$ & $\alpha/\nu$ & $\beta/\nu$ & $\gamma/\nu$ \\ \hline
	L0 & 0.5927 & 0.75  & 0.6799 & 0.103  & 0.64071  \\ \hline
	L1 & 0.5782 & 0.736 & 0.6712 & 0.1026 & 0.6287  \\ \hline
	L2 & 0.5701 & 0.721 & 0.6631 & 0.0982 & 0.6362  \\ \hline
\end{tabular}
\caption{List of combined exponents}
\label{tab:exponents-combined}
\end{table}
The critical exponents found in all cases are listed below,\\
\begin{table}
\centering
\begin{tabular}{|c|c|c|c|c|c|c|c|}
	\hline
	Interaction & $p_c$ & $1/\nu$ & $\alpha$ & $\beta$ & $\gamma$ & $\alpha+2\beta+\gamma$ & $d_f$\\ \hline
	L0 & 0.5927 & 0.75  & 0.906 & 0.137 & 0.8543 & 2.0347 & 1.8939  \\ \hline
	L1 & 0.5782 & 0.736 & 0.911 & 0.139 & 0.8542 & 2.044  & 1.8994  \\ \hline
	L2 & 0.5701 & 0.721 & 0.919 & 0.136 & 0.882  & 2.07   & 1.90810 \\ \hline

\end{tabular}
\caption{List of exponents}
\label{tab:exponents}
\end{table}
Here we notice that the critical point decreases as we increase the range of interaction. But the fractal dimension increases. This is reasonable since my occupying nearest and second nearest neighbor we are increasing the change of any individual cluster to grow faster. This is the reason for the $p_c$ value to decrease. But it grows in area not in length average meaning when the spanning cluster appears it will contain more sites and bonds than in regular percolation which is evident from the fractal dimension $d_f$. \\
All other exponents changes a bit but their shape is not different. That's why change is not visible to the naked eye and it requires a thorough investigation.