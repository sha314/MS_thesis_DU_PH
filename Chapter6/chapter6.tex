
\chapter{Summary and Discussion}
We have investigated percolation by random sequential ballistic deposition (RSBD) on a square lattice 
with interaction range upto second nearest neighbors. The critical points $p_c$ and all the necessary critical exponents $\alpha$, $\beta$, $\gamma$, $\nu$ etc. are obtained numerically for each range of interactions. Like  in its thermal counterpart, we find that the critical exponents of RSBD depend on the range of interactions and for a given range of interaction they obey the Rushbrooke inequality. Besides, we obtain the exponent $\tau$ which characterizes the cluster size distribution \ref{} and the fractal dimension $d_f$ that characterizes the spanning cluster at $p_c$. Our results suggest that the RSBD for each range of interaction belong to a new universality class which is in sharp contrast to earlier results of the only work that exhist on RSBD.\\

We denote $L0,L1,L2$ for expressing direct, first nearest neighbor and second nearest neighbor interaction respectively. Up until now we knew the exponents for $L0$ for old definition of site percolation. We have found same exponents for the thermodynamically consistent new definition of site percolation and we also have investigated for short and long range interactions and we call this $L1$ and $L2$ respectively. Note that $L1$ is the case where we can choose the first nearest neighbor and $L2$ is the case where we can chose 2nd nearest neighbor in the direction of first nearest neighbor. We can use new feature of $L1$ only if the feature of $L0$ is unavailable, i.e., the selected site is already occupied. Similarly we can use new feature of $L2$ only if the feature of $L0$ and $L1$ is unavailable. Using this in mind we perform simulation and we obtain the critical exponents which agree with the laws of thermodynamics and the Rushbrooke inequality is satisfied in all cases.\\
The combined exponents are listed below\\
\begin{table}
\centering
\begin{tabular}{|c|c|c|c|c|c|}
	\hline
	Interaction & $p_c$ & $1/\nu$ & $\alpha/\nu$ & $\beta/\nu$ & $\gamma/\nu$ \\ \hline
	L0 & 0.5927 & 0.75  & 0.6799 & 0.103  & 0.64071  \\ \hline
	L1 & 0.5782 & 0.736 & 0.6712 & 0.1026 & 0.6287  \\ \hline
	L2 & 0.5701 & 0.721 & 0.6631 & 0.0982 & 0.6362  \\ \hline
\end{tabular}
\caption{List of combined exponents}
\label{tab:exponents-combined}
\end{table}
The critical exponents found in all cases are listed below,\\
\begin{table}
\centering
\begin{tabular}{|c|c|c|c|c|}
	\hline
	Interaction & $\alpha$ & $\beta$ & $\gamma$ & $\alpha+2\beta+\gamma$ \\ \hline
	L0  & 0.906 & 0.137 & 0.8543 & 2.0347   \\ \hline
	L1  & 0.911 & 0.139 & 0.8542 & 2.044   \\ \hline
	L2  & 0.919 & 0.136 & 0.882  & 2.07    \\ \hline
\end{tabular}
\caption{Exponents Satisfying Rushbrooke Inequality}
\label{tab:rushbrooke}
\end{table}

\begin{table}
	\centering
	\begin{tabular}{|c|c|c|}
		\hline
		Interaction 	& $d_f$   & $\tau$   \\ \hline
		L0 (standard) 	& 91/48   & 187/91   \\ \hline
		L0 (obtained)	& 1.8939  & 2.0531 \\ \hline
		L1				& 1.8994  & 1.9771     \\ \hline
		L2				& 1.9081  & 1.9636   \\ \hline
	\end{tabular}
	\caption{Exponents giving cluster information}
	\label{tab:cluster-info}
\end{table}
		
Here we notice that the critical point decreases as we increase the range of interaction. But the fractal dimension increases. This is reasonable since my occupying nearest and second nearest neighbor we are increasing the change of any individual cluster to grow faster. This is the reason for the $p_c$ value to decrease. But it grows in area not in length average meaning when the spanning cluster appears it will contain more sites and bonds than in regular percolation which is evident from the fractal dimension $d_f$. \\
All other exponents changes a bit but their shape is not different. That's why change is not visible to the naked eye and it requires a thorough investigation.