%!TEX root = ../thesis.tex
%*******************************************************************************
%*********************************** First Chapter *****************************
%*******************************************************************************

\ifpdf
\graphicspath{{Chapter1/Figs/Raster/}{Chapter1/Figs/PDF/}{Chapter1/Figs/}}
\else
\graphicspath{{Chapter1/Figs/Vector/}{Chapter1/Figs/}}
\fi

\chapter{Introduction}  %Title of the First Chapter
%********************************** %First Section  **************************************
Percolation is one of the most studied problems in statistical physics. Its idea was first conceived by Paul Flory in the early 1940s in the context of gelation in polymers \cite{Flory1941}. Since then it has come a long way holding the hands of respected scientists. Physicists are always interested in describing natural phenomena in a simplified way. Phase transition is one of the most studied problem in physics and using percolation phenomena we can mimic phase transition in real system. Phase transition always includes at least one critical point below which the system behaves in one way and above it the system behaves in a completely different way. For example liquid to gas transition. Phase transition as of two kinds: first order and second order, which is classified in terms of discontinuity of derivatives of thermodynamic potentials.

%********************************** %Second Section  *************************************
\section{Motivation and Objective} %Section - 1.2

%********************************** % Third Section  *************************************
\section{Method of Study}  %Section - 1.3 

\section{Organization of Chapters}
