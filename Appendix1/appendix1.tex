%!TEX root = ../thesis.tex
% ******************************* Thesis Appendix A ****************************
\definecolor{mGreen}{rgb}{0,0.6,0}
\definecolor{mGray}{rgb}{0.5,0.5,0.5}
\definecolor{mPurple}{rgb}{0.58,0,0.82}
\definecolor{backgroundColour}{rgb}{0.95,0.95,0.92}

\lstdefinestyle{CStyle}{
	backgroundcolor=\color{backgroundColour},   
	commentstyle=\color{mGreen},
	keywordstyle=\color{magenta},
	numberstyle=\tiny\color{mGray},
	stringstyle=\color{mPurple},
	basicstyle=\footnotesize,
	breakatwhitespace=false,         
	breaklines=true,                 
	captionpos=b,                    
	keepspaces=true,                 
	numbers=left,                    
	numbersep=5pt,                  
	showspaces=false,                
	showstringspaces=false,
	showtabs=false,                  
	tabsize=2,
	language=C
}


\lstdefinestyle{CMakeStyle}{
	backgroundcolor=\color{backgroundColour},   
	basicstyle=\footnotesize,
	breakatwhitespace=false,         
	breaklines=true,                 
	captionpos=b,                    
	keepspaces=true,                 
	numbers=left,                    
	numbersep=5pt,                  
	showspaces=false,                
	showstringspaces=false,
	showtabs=false,                  
	tabsize=2,
}

\ifpdf
\graphicspath{{Appendix1/figs/}}
\else
\graphicspath{{Appendix1/figs/}}
\fi

\chapter{Percolation} 
\section{Algorithm}

	The UML diagram for the program is 
	\begin{figure}
		\centering
		\includegraphics[width=16cm]{{{ClassDiagram2}}}
		\caption{Schematic UML diagram for Site Percolation Ballistic Deposition program. This figure shows the dependencies and inheritance os the Classes and Structs in the program.}
	\end{figure}

\section{Code}
	
	Each header file starts with a directive \textit{\#ifndef} and \textit{\#define}, which is necessary because one header file is needed multiple times and including it more than once results in error. These directive prevents it. Of course this directive must be closed by \textit{\#endif}.
	
	
	\subsection{Index}
	Here the notion of index of site and index of bond is defined. Site index has two element which determine row and column. Bond Index had three element describing bond type, row, column. Bond type can be horizontal or vertical.
	%!TEX root = ../thesis.tex
% ******************************* Thesis Appendix A ****************************

the \textbf{src/index/index.h} file
\begin{lstlisting}[style=CStyle]
#ifndef SITEPERCOLATION_INDEX_H
#define SITEPERCOLATION_INDEX_H

#include <ostream>
#include <iostream>
#include <sstream>
#include <vector>

#include "../types.h"
#include "../exception/exceptions.h"
#include "../lattice/bond_type.h"


struct Index{
value_type row_{};
value_type column_{};

~Index()                      = default;
Index()                       = default;

Index(value_type x, value_type y) : row_{x}, column_{y} {}

};


class IndexRelative{
public:
int x_{};
int y_{};

~IndexRelative()                      = default;
IndexRelative()                       = default;

IndexRelative(int x, int y) : x_{x}, y_{y} {}

};

struct BondIndex{
BondType bondType;

value_type row_;
value_type column_;

~BondIndex()                        = default;
BondIndex()                         = default;

BondIndex(BondType hv, value_type row, value_type column)
:  row_{row}, column_{column}
{
bondType = hv;
}

bool horizontal() const { return bondType == BondType::Horizontal;}
bool vertical() const { return bondType == BondType::Vertical;}

};


std::ostream& operator<<(std::ostream& os, const Index& index);
bool operator==(const Index& index1, const Index& index2);
bool operator<(const Index& index1, const Index& index2);

std::ostream& operator<<(std::ostream& os, const IndexRelative& index);

std::ostream& operator<<(std::ostream& os, const BondIndex& index);
bool operator==(BondIndex index1, BondIndex index2);
bool operator<(BondIndex index1, BondIndex index2);


/**
*  Get 2nd nearest neightbor / sin the direction of 1st nearest neighbor, while @var center is the center
*/
Index get_2nn_in_1nn_direction(Index center, Index nn_1, value_type length);
std::vector<Index> get_2nn_s_in_1nn_s_direction(Index center, const std::vector<Index> &nn_1, value_type length);

#endif /* SITEPERCOLATION_INDEX_H */
\end{lstlisting}


The \textbf{src/index/index.cpp} file
\begin{lstlisting}[style=CStyle]
#include <iomanip>
#include "index.h"

using namespace std;

ostream& operator<<(ostream& os, const Index& index)
{
return os <<'(' << index.row_ << ',' << index.column_ << ')';
}

ostream& operator<<(ostream& os, const IndexRelative& index)
{
return os << '(' << std::setw(3) << index.x_ << ',' << std::setw(3) << index.y_ << ')';
}

bool operator==(const Index& index1, const Index& index2){
return (index1.row_ == index2.row_) && (index1.column_ == index2.column_);
}

bool operator<(const Index& index1, const Index& index2){
if(index1.row_ < index2.row_)
return true;
if(index1.row_ == index2.row_){
return index1.column_ < index2.column_;
}
return false;
}

ostream& operator<<(ostream& os, const BondIndex& index){
if(index.horizontal()){
// horizontal bond
os << "<->" ;
}
if (index.vertical()){
// vertical bond
os << "<|>" ;
}
return os << '(' << index.row_ << ',' << index.column_ << ')';
}

bool operator==(BondIndex index1, BondIndex index2){
if(index1.horizontal() ==  index2.horizontal() || index1.vertical() ==  index2.vertical()){
// horizontal or vertical
return index1.row_ == index2.row_ && index1.column_ == index2.column_;
}
return false;
}


bool operator<(BondIndex index1, BondIndex index2){
cout << "not yet defined : line " << __LINE__ << endl;
return false;
}


/**
* Get the 2nd nearest nearest neighbor in the direction of 1st nearest neighbor.
* Periodicity is not considered here.
*/
Index get_2nn_in_1nn_direction(Index center, Index nn_1, value_type length){
int delta_c = int(nn_1.column_) - int(center.column_);
int delta_r = int(nn_1.row_) - int(center.row_);
if (delta_c == 0 && delta_r == 0){
cout << "Both indices are same : line " << __LINE__ << endl;
}
else if(delta_c > 1 || delta_r > 1){
// meaning, the sites are on the opposite edges
//        cout << "2nd index is not the First nearest neighbor : line " << __LINE__ << " : file " << __FILE__ << endl;
}

return Index{(nn_1.row_ + delta_r + length) % length, (nn_1.column_ + delta_c + length) % length};
}


/**
* Get all second nearest neighbors based on the first nearest neighbors.
* Periodicity is not considered here
*/
vector<Index> get_2nn_s_in_1nn_s_direction(Index center, const vector<Index> &nn_1, value_type length){
vector<Index> nn_2(nn_1.size());

for(size_t i{}; i != nn_1.size() ; ++i){
int delta_c = int(nn_1[i].column_) - int(center.column_);
int delta_r = int(nn_1[i].row_) - int(center.row_);
if (delta_c == 0 && delta_r == 0){
cout << "Both indices are same : line " << __LINE__ << endl;
}
else if(delta_c > 1 || delta_r > 1){
// meaning, the sites are on the opposite edges
//            cout << "center " << center << " nn " << nn_1 << endl;
//            cout << "2nd index is not the First nearest neighbor : line " << __LINE__ << " : file " << __FILE__ << endl;
}

nn_2[i] =  Index{(nn_1[i].row_ + delta_r + length) % length, (nn_1[i].column_ + delta_c + length) % length};
}
return nn_2;
}
\end{lstlisting}

	
	
	
	\subsection{Site}
	The site class contains all information about a site, e.g., if it is active or not and if it is then what is it's group id or relative index.
	%!TEX root = ../thesis.tex
% ******************************* Thesis Appendix A ****************************

The \textbf{src/lattice/site.h} file

\begin{lstlisting}[style=CStyle]
struct Site{
bool _status{false};
int _group_id{-1};
Index _id{};
IndexRelative _relative_index{0,0};
public:
~Site()                 = default;
Site()                  = default;
Site(const Site&)             = default;
Site(Site&&)            = default;
Site& operator=(const Site&)  = default;
Site& operator=(Site&&) = default;
Site(Index id, value_type length){
if(id.row_ >= length || id.column_ >= length){
std::cout << "out of range : line " << __LINE__ << std::endl;
}
_id.row_ = id.row_;
_id.column_ = id.column_;
}

bool isActive() const { return _status;}
void activate(){ _status = true;}
void deactivate() {
_relative_index = {0,0};
_group_id = -1;
_status = false;
}
Index ID() const { return  _id;}

int     get_groupID() const {return _group_id;}
void    set_groupID(int g_id) {_group_id = g_id;}
std::stringstream getSite() const {
std::stringstream ss;
if(isActive())
ss << _id;
else
ss << "(*)";
return ss;
}
void relativeIndex(IndexRelative r){
_relative_index = r;
}
void relativeIndex(int x, int y){
_relative_index = {x,y};
}
IndexRelative relativeIndex() const {return _relative_index;}
};
std::ostream& operator<<(std::ostream& os, const Site& site);
bool operator==(Site& site1, Site& site2);
\end{lstlisting}

	
	\subsection{Bond}
	The bond class contains all information about a bond, e.g., if it is active or not and if it is then what is it's group id.
	%!TEX root = ../thesis.tex
% ******************************* Thesis Appendix A ****************************

The \textbf{src/lattice/bond.h} file

\begin{lstlisting}[style=CStyle]
struct Bond{
bool _status{false};
value_type _length;
int _group_id{-1};
BondType bondType;
IndexRelative _relative_index{0,0};
Index _end1;
Index _end2;
BondIndex _id;

~Bond() = default;
Bond() = default;
Bond(Index end1, Index end2, value_type length){
_end1.row_ = end1.row_;
_end1.column_ = end1.column_;
_end2.row_ = end2.row_;
_end2.column_ = end2.column_;
_length = length;
if(_end1.row_ == _end2.row_){
bondType = BondType::Horizontal;
if(_end1.column_ > _end2.column_){
if(_end1.column_ == _length-1 && _end2.column_ ==0){
}
else{
_end1.column_ = end2.column_;
_end2.column_ = end1.column_;
}
}
else if(_end1.column_ < _end2.column_){
if(_end1.column_ == 0 && _end2.column_ == _length-1){
_end1.column_ = end2.column_;
_end2.column_ = end1.column_;
}
}
}
else if(_end1.column_ == _end2.column_){
bondType = BondType::Vertical;
if(_end1.row_ > _end2.row_){
if(_end1.row_ == _length-1 && _end2.row_ ==0){
}
else{
_end1.row_ = end2.row_;
_end2.row_ = end1.row_;
}
}
else if(_end1.row_ < _end2.row_){
if(_end1.row_ == 0 && _end2.row_ == _length-1){
_end1.row_ = end2.row_;
_end2.row_ = end1.row_;
}
}
}
else{
std::cout << '(' << _end1.row_ << ',' << _end1.column_ << ')' << "<->"
<< '(' << _end2.row_ << ',' << _end2.column_ << ')'
<< " is not a valid bond : line " << __LINE__ << std::endl;
}
_id = BondIndex(bondType, _end1.row_, _end1.column_);  // unsigned long
}

std::vector<Index> getSites() const { return {_end1, _end2};}
Index id() const {return _end1;}
BondIndex ID() const {return _id;}
void activate() {_status = true;}
void deactivate() {
_relative_index = {0,0};
_group_id = -1;
_status = false;
}
bool isActive() const { return _status;}

int get_groupID() const {return _group_id;}
void set_groupID(int g_id) {_group_id = g_id;}

bool isHorizontal() const { return bondType == BondType ::Horizontal;}
bool isVertical()   const { return bondType == BondType ::Vertical;}
void relativeIndex(IndexRelative r){
_relative_index = r;
}
void relativeIndex(int x, int y){
_relative_index = {x,y};
}
IndexRelative relativeIndex() const {return _relative_index;}
};
\end{lstlisting}

The \textbf{src/lattice/bond\_type.h} file
\begin{lstlisting}[style=CStyle]
enum class BondType{
Horizontal,
Vertical
};
\end{lstlisting}
			
	\subsection{Lattice}
	The lattice class consists of array of sites and bonds. This class contains information about lattice size. And contains functions to view the lattice differently in the console.
	%!TEX root = ../thesis.tex
% ******************************* Thesis Appendix A ****************************

the \textbf{src/lattice/lattice.h} file
\begin{lstlisting}[style=CStyle]
class SqLattice {
std::vector<std::vector<Site>> _sites;  // holds all the sites
std::vector<std::vector<Bond>> _h_bonds;  // holds all horizontal bonds
std::vector<std::vector<Bond>> _v_bonds;  // holds all vertical bonds
bool _bond_resetting_flag=true; // so that we can reset all bonds
bool _site_resetting_flag=true; // and all sites
value_type _length{};
private:
void reset_sites();
void reset_bonds();
public:
~SqLattice() = default;
SqLattice() = default;
SqLattice(SqLattice&) = default;
SqLattice(SqLattice&&) = default;
SqLattice& operator=(const SqLattice&) = default;
SqLattice& operator=(SqLattice&&) = default;
SqLattice(value_type length, bool activate_bonds, bool activate_sites, bool bond_reset, bool site_reset);
void reset(bool reset_all=false);
void activate_site(Index index);
void activateBond(BondIndex bond);
void deactivate_site(Index index);
void deactivate_bond(Bond bond);
value_type length() const { return  _length;}
Site& getSite(Index index);
Bond& getBond(BondIndex);
const Site& getSite(Index index) const ;
const Bond& getBond(BondIndex index) const ;
void setGroupID(Index index, int group_id);
void setGroupID(BondIndex index, int group_id);
const int getGroupID(Index index)const;
const int getGroupID(BondIndex index)const;
std::vector<Index> get_neighbor_site_indices(Index site);   // site neighbor of site
std::vector<BondIndex> get_neighbor_bond_indices(BondIndex site); // bond neighbor of bond
std::vector<Index> get_neighbor_indices(BondIndex bond);   // two site neighbor of bond.
static std::vector<Index> get_neighbor_site_indices(size_t length, Index site);   // 4 site neighbor of site
static std::vector<BondIndex> get_neighbor_bond_indices(size_t length, BondIndex site); // 6 bond neighbor of bond
static std::vector<Index> get_neighbor_indices(size_t length, BondIndex bond);   // 2 site neighbor of bond.
};
\end{lstlisting}




	
	\subsection{Exception}
	Different types of exceptions.
	%!TEX root = ../thesis.tex
% ******************************* Thesis Appendix A ****************************

The \textbf{src/exception/exception.h} file

\begin{lstlisting}[style=CStyle]
struct Mismatch{
    std::string msg_;
    size_t line_;
    Mismatch(size_t line, std::string msg="")
            :line_{line}, msg_{msg}  {}

    void what() const {
        std::cerr << msg_ << "\nId and index mismatch at line " << line_ << std::endl;
    }
};

struct InvalidIndex{
    std::string msg_;
    InvalidIndex(std::string msg)  :msg_{msg}  {}

    void what() const {
        std::cerr << msg_ << std::endl;
    }
};

struct InvalidBond{
    std::string msg_;
    InvalidBond(std::string msg)  :msg_{msg}  {}

    void what() const {
        std::cout << msg_ << std::endl;
    }
};

struct OccupiedNeighbor{
    std::string msg_;
    OccupiedNeighbor(std::string msg): msg_{msg}{}

    void what() const {
        std::cout << msg_ << std::endl;
    }
};

struct InvalidNeighbor{
    std::string msg_;
    InvalidNeighbor(std::string msg)  :msg_{msg}  {}
    void what() const {
        std::cout << msg_ << std::endl;
    }
};
\end{lstlisting}



	
	\subsection{Cluster}
	The cluster class contains all information about a cluster. Number of sites and bonds in a cluster and the id of the cluster is contained in a cluster.
	%!TEX root = ../thesis.tex
% ******************************* Thesis Appendix A ****************************

The \textbf{src/percolation/cluster.h} file

\begin{lstlisting}[style=CStyle]
#ifndef SITEPERCOLATION_CLUSTER_H
#define SITEPERCOLATION_CLUSTER_H

#include <vector>
#include <set>
#include "../lattice/bond.h"
#include "../types.h"
#include "../lattice/site.h"


/**
* Cluster of bonds and sites
* version 3
* final goal -> make a template cluster. so that we can use it for Bond cluster or Site cluster
* root site (bond) is the first site (bond) of the cluster. nedeed for (wrapping) site percolation
*/
class Cluster{
// contains bond and site
std::vector<BondIndex>  _bond_index; // BondIndex for indexing bonds
std::vector<Index>      _site_index; // Site index

int _creation_time{-1};  // holds the creation birthTime of a cluster object
int _id{-1};
public:
//    using iterator = std::vector<Bond>::iterator;

~Cluster()                           = default;
Cluster()                            = default;
Cluster(Cluster&)                 = default;
Cluster(Cluster&&)                = default;
Cluster& operator=(const Cluster&)      = default;
Cluster& operator=(Cluster&&)     = default;

explicit Cluster(int id){

_id = id;       // may be modified in the program

/*
* Only readable, not modifiable.
* when time = 0 => only lattice exists and bonds in site percolation, not any sites
* When id = 0, time = 1 => we have placed the first site, hence created a cluster with size greater than 1
*      Only then Cluster constructor is called.
*
*/
_creation_time = id + 1;       // only readable, not modifiable
}


void addSiteIndex(Index );
void addBondIndex(BondIndex );

Index lastAddedSite(){return _site_index.back();}
BondIndex lastAddedBond(){return _bond_index.back();}


void insert(const std::vector<BondIndex>& bonds);
void insert(const std::vector<Index>& sites);

void insert(const Cluster& cluster);
void insert_v2(const Cluster& cluster);
void insert_with_id_v2(const Cluster& cluster, int id);


friend std::ostream& operator<<(std::ostream& os, const Cluster& cluster);

const std::vector<BondIndex>&   getBondIndices()    {return _bond_index;}
const std::vector<Index>&       getSiteIndices()    {return _site_index;}

const std::vector<BondIndex>&   getBondIndices()  const  {return _bond_index;}
const std::vector<Index>&       getSiteIndices()  const  {return _site_index;}

value_type numberOfBonds() const { return _bond_index.size();}
value_type numberOfSites() const { return _site_index.size();}
int get_ID() const { return _id;}
void set_ID(int id) { _id = id;}

int birthTime() const {return _creation_time;}

Index getRootSite()const{return _site_index[0];} // for site percolation
BondIndex getRootBond()const{return _bond_index[0];} // for bond percolation
bool empty() const { return _bond_index.empty() && _site_index.empty();}
void clear() {_bond_index.clear(); _site_index.clear(); }
};

#endif //SITEPERCOLATION_CLUSTER_H
\end{lstlisting}

The \textbf{src/percolation/cluster.cpp} file

\begin{lstlisting}[style=CStyle]

#include "cluster.h"

using namespace std;

// add Site index
void Cluster::addSiteIndex(Index index) {
_site_index.push_back(index);
}

void Cluster::addBondIndex(BondIndex bondIndex) {
_bond_index.push_back(bondIndex);
}



void Cluster::insert(const std::vector<BondIndex>& bonds){
_bond_index.reserve(bonds.size());
for(value_type i{} ; i != bonds.size() ; ++i){
_bond_index.push_back(bonds[i]);
}
}

void Cluster::insert(const std::vector<Index>& sites){
_site_index.reserve(sites.size());
for(value_type i{} ; i != sites.size() ; ++i){
_site_index.push_back(sites[i]);
}
}

/**
* Merge two cluster as one
* All intrinsic property should be considered, e.g., creation time of a cluster must be recalculated
* @param cluster
*/
void Cluster::insert(const Cluster &cluster) {
if(_id > cluster._id){
cout << "_id > cluster._id : line " << __LINE__ << endl;
_id = cluster._id;
}
// older time or smaller time is the creation birthTime of the cluster
//    cout << "Comparing " << _creation_time << " and " << cluster._creation_time;
_creation_time = _creation_time < cluster._creation_time ? _creation_time : cluster._creation_time;
//    cout << " Keeping " << _creation_time << endl;
_bond_index.insert(_bond_index.end(), cluster._bond_index.begin(), cluster._bond_index.end());
_site_index.insert(_site_index.end(), cluster._site_index.begin(), cluster._site_index.end());
}


/**
* Merge two cluster as one
* All intrinsic property should be considered, e.g., creation time of a cluster must be recalculated
* @param cluster
*/
void Cluster::insert_v2(const Cluster &cluster) {
// older time or smaller time is the creation birthTime of the cluster
//    cout << "Comparing " << _creation_time << " and " << cluster._creation_time;
_creation_time = _creation_time < cluster._creation_time ? _creation_time : cluster._creation_time;
//    cout << " Keeping " << _creation_time << endl;
_bond_index.insert(_bond_index.end(), cluster._bond_index.begin(), cluster._bond_index.end());
_site_index.insert(_site_index.end(), cluster._site_index.begin(), cluster._site_index.end());
}


void Cluster::insert_with_id_v2(const Cluster &cluster, int id) {
_id = id;
// older time or smaller time is the creation birthTime of the cluster
//    cout << "Comparing " << _creation_time << " and " << cluster._creation_time;
_creation_time = _creation_time < cluster._creation_time ? _creation_time : cluster._creation_time;
//    cout << " Keeping " << _creation_time << endl;
_bond_index.insert(_bond_index.end(), cluster._bond_index.begin(), cluster._bond_index.end());
_site_index.insert(_site_index.end(), cluster._site_index.begin(), cluster._site_index.end());
}


std::ostream &operator<<(std::ostream &os, const Cluster &cluster) {
os << "Sites : size (" << cluster._site_index.size() << ") : ";
os << '{';
for(auto a: cluster._site_index){
os << a << ',';
}
os << '}' << endl;

os << "Bonds : size (" << cluster._bond_index.size() <<") : ";
os << '{';
for(auto a: cluster._bond_index){
os << a << ',';
}
os << '}';

return os << endl;
}
\end{lstlisting}
	
	\subsection{Percolation}
	The \textit{SqLatticePercolation} class contains generic operation that to performed for percolation on square lattice. It's subclass  \textit{SitePercolation\_ps\_v9} is the class when all required method for general site percolation with our definition is defined. 
	And it's subclass \textit{SitePercolationBallisticDeposition\_v2}
	contains some method for ballistic deposition for $l=\{1,2\}$ which extends to two new subclass \textit{SitePercolationBallisticDeposition\_L1\_v2} and \textit{SitePercolationBallisticDeposition\_L2\_v2} with detailed method for ballistic deposition $l=1$ and $l=2$ respectively.
	
	%!TEX root = ../thesis.tex
% ******************************* Thesis Appendix A ****************************

The \textbf{src/percolation/percolation.h} file

\begin{lstlisting}[style=CStyle]
class SqLatticePercolation{
value_type  _length;
value_type _max_number_of_bonds;
value_type _max_number_of_sites;
char type{'0'}; // percolation type. 's' -> site percolation. 'b' -> bond percolation
protected:
SqLattice _lattice;
value_type _index_sequence_position{};

std::vector<Cluster> _clusters;   
double _occuption_probability {};
double _entropy{};
double _entropy_current{};
size_t _cluster_count{};
value_type _bonds_in_cluster_with_size_two_or_more{0};
bool _reached_critical = false; 
value_type _total_relabeling{};
double time_relabel{};
value_type _number_of_occupied_sites{};
value_type _max_iteration_limit{};
std::random_device _random_device;
std::mt19937 _random_generator;
void set_type(char t){type = t;} 
public:
static constexpr const char* signature = "SqLatticePercolation";
virtual ~SqLatticePercolation() = default;
SqLatticePercolation(value_type length);
void reset();
bool occupy();
value_type length() const { return _length;}
value_type maxSites() const {return _max_number_of_sites;}
value_type maxBonds() const { return _max_number_of_bonds;}

virtual double occupationProbability() const { return _occuption_probability;}
virtual double entropy() { return _entropy_current;}
double entropy_by_site(); // for future convenience. // the shannon entropy. the full calculations. time consuming
double entropy_by_bond(); // for future convenience. // the shannon entropy. the full calculations. time consuming
size_t numberOfcluster() const {return _cluster_count;}

void get_cluster_info(
std::vector<value_type> &site,
std::vector<value_type> &bond
);

char get_type() const {return type;} // get percolation type
virtual value_type maxIterationLimit() {return _max_iteration_limit;};

double get_relabeling_time() const {return time_relabel;}
value_type relabeling_count() const {return _total_relabeling;}
};

/**
* Site Percolation by Placing Sites
* version 9
*/
class SitePercolation_ps_v9 : public SqLatticePercolation{
protected:
bool _periodicity{false};

value_type min_index; // minimum index = 0
value_type max_index; // maximum index = length - 1

// index sequence
std::vector<Index> index_sequence;  
std::vector<value_type> randomized_index;
int _cluster_id{};
value_type _index_last_modified_cluster{}; 
value_type _number_of_bonds_in_the_largest_cluster{};
value_type _number_of_sites_in_the_largest_cluster{};   

Index _last_placed_site;    // keeps track of last placed site

std::vector<Index> _top_edge, _bottom_edge, _left_edge, _right_edge;
std::vector<Index> _spanning_sites;
std::vector<Index> _wrapping_sites;
std::vector<value_type> number_of_sites_to_span;
std::vector<value_type> number_of_bonds_to_span;
value_type _total_relabeling{};
void relabel_sites(const std::vector<Index> &sites, int id_a, int delta_x_ab, int delta_y_ab) ;
double time_relabel{};
public:
static constexpr const char* signature = "SitePercolation_ps_v8";

~SitePercolation_ps_v9() = default;
SitePercolation_ps_v9() = default;
SitePercolation_ps_v9(SitePercolation_ps_v9 & ) = default;
SitePercolation_ps_v9(SitePercolation_ps_v9 && ) = default;
explicit SitePercolation_ps_v9(value_type length, bool periodicity=true);

SitePercolation_ps_v9& operator=(SitePercolation_ps_v9 & ) = default;
double get_relabeling_time() {return time_relabel;}
value_type relabeling_count() const {return _total_relabeling;}
virtual void reset();
bool periodicity() const {return _periodicity;}
std::string getSignature();
void add_entropy_for_bond(value_type index);
void subtract_entropy_for_bond(const std::set<value_type> &found_index_set, int base=-1);
virtual bool occupy();
value_type placeSite_weighted(Index site); // uses weighted relabeling by first identifying the largest cluster
value_type placeSite_weighted(Index site,
std::vector<Index>& neighbor_sites,
std::vector<BondIndex>& neighbor_bonds);
Index selectSite(); // selecting site
void connection_v2(Index site, std::vector<Index> &site_neighbor, std::vector<BondIndex> &bond_neighbor);
void relabel_sites_v5(Index root_a, const Cluster& clstr_b); // relative index is set accordingly
double numberOfOccupiedSite() const { return _number_of_occupied_sites;}
double occupationProbability() const { return double(_number_of_occupied_sites)/maxSites();}
double entropy(); // the shannon entropy
value_type numberOfBondsInTheLargestCluster_v2();
value_type numberOfSitesInTheLargestCluster();
value_type numberOfSitesInTheSpanningClusters_v2()  ;
value_type numberOfBondsInTheSpanningClusters_v2()  ;
value_type numberOfSitesInTheWrappingClusters()  ;
value_type numberOfBondsInTheWrappingClusters()  ;
bool detectSpanning_v6(const Index& site);
bool check_if_id_matches(Index site, const std::vector<Index> &edge);
bool detectWrapping();
void track_numberOfBondsInLargestCluster();
void track_numberOfSitesInLargestCluster();
Index lastPlacedSite() const { return _last_placed_site;}
void spanningIndices() const;
void wrappingIndices() const;
void writeVisualLatticeData(const std::string& filename, bool only_spanning=true);
protected:
void initialize();
void initialize_index_sequence();
void randomize_v2(); // better random number generator

int find_cluster_index_for_placing_new_bonds(const std::vector<Index> &neighbors, std::set<value_type> &found_indices);

value_type manage_clusters(
const std::set<value_type> &found_index_set,
std::vector<BondIndex> &hv_bonds,
Index &site,
int base_id // since id and index is same
);
public:
IndexRelative getRelativeIndex(Index root, Index site_new);
};


class SitePercolationBallisticDeposition_v2: public SitePercolation_ps_v9{
protected:
std::vector<value_type> indices;
std::vector<value_type> indices_tmp;
public:
static constexpr const char* signature = "SitePercolation_BallisticDeposition_v2";
virtual ~SitePercolationBallisticDeposition_v2(){
indices.clear();
indices_tmp.clear();
};
SitePercolationBallisticDeposition_v2(value_type length, bool periodicity);
virtual bool occupy();

Index select_site(std::vector<Index> &sites, std::vector<BondIndex> &bonds);
Index select_site_upto_1nn(std::vector<Index> &sites, std::vector<BondIndex> &bonds);
Index select_site_upto_2nn(std::vector<Index> &sites, std::vector<BondIndex> &bonds);

void reset();
void initialize_indices();

virtual std::string getSignature() {
std::string s = "sq_lattice_site_percolation_ballistic_deposition_";
if(_periodicity)
s += "_periodic_";
else
s += "_non_periodic_";
return s;
}
value_type placeSite_1nn_v2();
value_type placeSite_2nn_v1();
};

class SitePercolationBallisticDeposition_L1_v2: public SitePercolationBallisticDeposition_v2{
public:
~SitePercolationBallisticDeposition_L1_v2() = default;
SitePercolationBallisticDeposition_L1_v2(value_type length, bool periodicity)
:SitePercolationBallisticDeposition_v2(length, periodicity){}

bool occupy() {
if(_number_of_occupied_sites == maxSites()){
return false;
}
try {
value_type v = placeSite_1nn_v2();
_occuption_probability = occupationProbability();
return v != ULLONG_MAX;
}catch (OccupiedNeighbor& on){
return false;
}
}

std::string getSignature() {
std::string s = "sq_lattice_site_percolation_ballistic_deposition_L1";
if(_periodicity)
s += "_periodic_";
else
s += "_non_periodic_";
return s;
}
};

class SitePercolationBallisticDeposition_L2_v2: public SitePercolationBallisticDeposition_v2{
public:
~SitePercolationBallisticDeposition_L2_v2() = default;
SitePercolationBallisticDeposition_L2_v2(value_type length, bool periodicity)
:SitePercolationBallisticDeposition_v2(length, periodicity){}

bool occupy() {
if(_number_of_occupied_sites == maxSites()){
return false;
}
try {
value_type v = placeSite_2nn_v1();
_occuption_probability = occupationProbability(); 
return v != ULLONG_MAX;
}catch (OccupiedNeighbor& on){
        on.what();
return false;
}
}
std::string getSignature() {
std::string s = "sq_lattice_site_percolation_ballistic_deposition_L2";
if(_periodicity)
s += "_periodic_";
else
s += "_non_periodic_";
return s;
}
};
\end{lstlisting}

The \textbf{src/percolation/percolation.cpp} file

\begin{lstlisting}[style=CStyle]
SqLatticePercolation::SqLatticePercolation(value_type length) {
if (length <= 2) {
/*
* Because if _length=2
* there are total of 4 distinct bond. But it should have been 8, i.e, (2 * _length * _length = 8)
*/
cerr << "_length <= 2 does not satisfy _lattice properties for percolation : line" << __LINE__ << endl;
exit(1);
}
_length = length;
value_type _length_squared = length * length;
_max_number_of_bonds = 2*_length_squared;
_max_number_of_sites = _length_squared;
_clusters = vector<Cluster>();

//    size_t seed = 0;
//    cerr << "automatic seeding is commented : line " << __LINE__ << endl;
auto seed = _random_device();
_random_generator.seed(seed); // seeding
cout << "seeding with " << seed << endl;
}


void
SqLatticePercolation::get_cluster_info(
vector<value_type> &site,
vector<value_type> &bond
) {
value_type total_site{}, total_bond{};
site.clear();
bond.clear();

unsigned long size = _clusters.size();
site.reserve(size);
bond.reserve(size);

value_type s, b;

for(value_type i{}; i < size; ++i){
if(_clusters[i].empty()){
//            cout << "Empty cluster : line " << endl;
continue;
}
s = _clusters[i].numberOfSites();
b = _clusters[i].numberOfBonds();
site.push_back(s);
bond.push_back(b);
total_site += s;
total_bond += b;
}
if(site.size() != bond.size()){
cout << "Size mismatched : line " << __LINE__ << endl;
}
//    cout << "total bonds " << total_bond << endl;
//    cout << "tatal sites " << total_site << endl;
if(type == 's'){
for(value_type j{total_bond}; j < maxBonds(); ++j){
bond.push_back(1); // cluster of length 1
total_bond += 1;
}
}
if(type == 'b'){
for(value_type j{total_site}; j < maxSites(); ++j){
total_site += 1;
site.push_back(1); // cluster of length 1
}
}
}

void SqLatticePercolation::reset() {
_lattice.reset();
_clusters.clear();
_index_sequence_position = 0;

_occuption_probability = 0;
// entropy
_entropy=0;
_entropy_current=0;
_total_relabeling = 0;
time_relabel = 0;
_cluster_count = 0;
_reached_critical = false;
}


/**
* Entropy calculation is performed here. The fastest method possible.
* Cluster size is measured by site.
* @return current entropy of the lattice
*/
double SqLatticePercolation::entropy_by_site() {
double H{}, mu ;

for(size_t i{}; i < _clusters.size(); ++i){
if(!_clusters[i].empty()){
mu = _clusters[i].numberOfSites() / double(_number_of_occupied_sites);
H += mu*log(mu);
}
}

return -H;
}

/**
* Entropy calculation is performed here. The fastest method possible.
* Cluster size is measured by site.
* @return current entropy of the lattice
*/
double SqLatticePercolation::entropy_by_bond() {
double H{}, mu ;

for(size_t i{}; i < _clusters.size(); ++i){
if(!_clusters[i].empty()){
mu = _clusters[i].numberOfBonds() / double(maxBonds());
H += mu*log(mu);
}
}

double number_of_cluster_with_size_one = maxBonds() - _bonds_in_cluster_with_size_two_or_more;
//    cout << " _bonds_in_cluster_with_size_two_or_more " << _bonds_in_cluster_with_size_two_or_more << " : line " << __LINE__ << endl;
mu = 1.0/double(maxBonds());
H += number_of_cluster_with_size_one * log(mu) * mu;

return -H;
}
\end{lstlisting}
	%!TEX root = ../thesis.tex
% ******************************* Thesis Appendix A ****************************

The \textbf{src/percolation/percolation\_site\_v9.cpp} file

\begin{lstlisting}[style=CStyle]
value_type SitePercolation_ps_v9::manage_clusters(
const set<value_type> &found_index_set,
vector<BondIndex> &hv_bonds,
Index &site,
int base_id
)
{
if (base_id != -1) {
value_type base = value_type(base_id);
_clusters[base].addSiteIndex(site);
int id_base = _clusters[base].get_ID();
vector<Index> neibhgors = _lattice.get_neighbor_site_indices(site);

IndexRelative r;
for(auto n: neibhgors){
if(_lattice.getGroupID(n) == id_base){
r = getRelativeIndex(n, site);
break; 
}
}

_clusters[base].insert(hv_bonds);
_lattice.getSite(site).relativeIndex(r);
_lattice.setGroupID(site, id_base); 
for(value_type ers: found_index_set){
_total_relabeling += _clusters[ers].numberOfSites(); // only for debugging purposes
relabel_sites_v5(site, _clusters[ers]);
 cluster
_clusters[base].insert_v2(_clusters[ers]);
_cluster_count--; // reducing number of clusters
_clusters[ers].clear(); // emptying the cluster
}
_index_last_modified_cluster = base;
} else {
_clusters.push_back(Cluster(_cluster_id));
value_type _this_cluster_index = _clusters.size() -1;
_lattice.setGroupID(site, _cluster_id); 
_cluster_count++; 
_cluster_id++;
_clusters.back().insert(hv_bonds);
_clusters[_this_cluster_index].addSiteIndex(site);
_index_last_modified_cluster = _this_cluster_index;   // last cluster is the place where new bonds are placed

}
return _index_last_modified_cluster;
}

bool SitePercolation_ps_v9::occupy() {
if(_index_sequence_position >= maxSites()){
return false;
}
Index site = selectSite();
placeSite_weighted(site);
_occuption_probability = occupationProbability(); // for super class
return true;
}

value_type SitePercolation_ps_v9::placeSite_weighted(Index current_site) {
// randomly choose a site
if (_number_of_occupied_sites == maxSites()) {
return ULONG_MAX; value
}
_last_placed_site = current_site;
_lattice.activate_site(current_site);
++_number_of_occupied_sites;
vector<BondIndex> bonds;
vector<Index>     sites;
connection_v2(current_site, sites, bonds);
_bonds_in_cluster_with_size_two_or_more += bonds.size();
set<value_type> found_index_set;
int  base_id = find_cluster_index_for_placing_new_bonds(sites, found_index_set);
subtract_entropy_for_bond(found_index_set, base_id); 
value_type merged_cluster_index = manage_clusters(
found_index_set, bonds, current_site, base_id
);
add_entropy_for_bond(merged_cluster_index);
track_numberOfBondsInLargestCluster();
track_numberOfSitesInLargestCluster();
return merged_cluster_index;
}

Index SitePercolation_ps_v9::selectSite(){
//    Index current_site = randomized_index_sequence[_index_sequence_position]; // old
value_type index = randomized_index[_index_sequence_position];
Index current_site = index_sequence[index]; // new process
++_index_sequence_position;
return current_site;
}


bool SitePercolation_ps_v9::detectWrapping() {
Index site = lastPlacedSite();
// only possible if the cluster containing 'site' has sites >= length of the lattice
if(_number_of_occupied_sites < length()){
return false;
}
if(_reached_critical){
return true; // reached critical in previous step
}
// check if it is already a wrapping site
int id = _lattice.getGroupID(site);
int tmp_id{};
for (auto i: _wrapping_sites){
tmp_id = _lattice.getGroupID(i);
if(id == tmp_id ){
return true;
}
}
vector<Index> sites = _lattice.get_neighbor_site_indices(site);
if(sites.size() < 2){ // at least two neighbor of  site is required
return false;
}else{
IndexRelative irel = _lattice.getSite(site).relativeIndex();
IndexRelative b;
for (auto a:sites){
if(_lattice.getGroupID(a) != _lattice.getGroupID(site)){
continue;
}
// belongs to the same cluster
b = _lattice.getSite(a).relativeIndex();
if(abs(irel.x_ - b.x_) > 1 || abs(irel.y_ - b.y_) > 1){
_wrapping_sites.push_back(site);
_reached_critical = true;
return true;
}
}
}
return !_wrapping_sites.empty();
}

\end{lstlisting}

The \textbf{src/percolation/percolation\_site\_ballistic\_deps\_v2.cpp} file

\begin{lstlisting}[style=CStyle]
bool SitePercolationBallisticDeposition_v2::occupy() {
if(_number_of_occupied_sites == maxSites()){
return false;
}
try {
value_type v = placeSite_1nn_v2();
_occuption_probability = occupationProbability();
return v != ULLONG_MAX;
}catch (OccupiedNeighbor& on){
        on.what();
return false;
}
}
\end{lstlisting}
	
	\subsection{Utilities}
	We sometimes need to print an array, std::map, std::vector in the console. Some functions for this task is given here. 
	%!TEX root = ../thesis.tex
% ******************************* Thesis Appendix A ****************************

The \textbf{src/util/printer.h} file

\begin{lstlisting}[style=CStyle]
#ifndef PERCOLATION_PRINTER_H
#define PERCOLATION_PRINTER_H

#include <ostream>
#include <unordered_map>
#include <set>
#include <unordered_set>
#include <vector>
#include <map>
#include <initializer_list>

template <typename T>
std::ostream& operator<<(std::ostream& os, const std::vector<T> & vec){
os << '{';
for(auto a: vec){
os << a << ',';
}
return os << '}';
}

template <typename T>
std::ostream& operator<<(std::ostream& os, const std::set<T> & vec){
os << '{';
for(auto a: vec){
os << a << ',';
}
return os << '}';
}

template <typename T>
std::ostream& operator<<(std::ostream& os, const std::unordered_set<T> & vec){
os << '{';
for(auto a: vec){
os << a << ',';
}
return os << '}';
}

template <typename K, typename V>
std::ostream& operator<<(std::ostream& os, const std::map<K, V> & m){
os << '{';
for(auto a: m){
os << '(' << a.first << "->" << a.second << "),";
}
return os << '}';
};

template <typename K, typename V>
std::ostream& operator<<(std::ostream& os, const std::unordered_map<K, V> & m){
os << '{';
for(auto a: m){
os << '(' << a.first << "->" << a.second << "),";
}
return os << '}';
};

/**
*
* Prints a horizontal barrier in the console.
* @param n             : how many time the middle string is repeated.
* @param initial       : string that is printed initially.
* @param middles       : middle string.
* @param end           : string that is printed at the end.
*/
void print_h_barrier(size_t n, const std::string& initial, const std::string& middles, const std::string& end="\n");

#endif //PERCOLATION_PRINTER_H
\end{lstlisting}

The \textbf{src/util/printer.cpp} file

\begin{lstlisting}[style=CStyle]
#include "printer.h"
#include <iostream>

using namespace std;

void print_h_barrier(size_t n, const string& initial, const string& middles, const string& end){
	cout << initial;
	for(size_t i{}; i < n ; ++i){
		cout << middles;
	}
	cout << end; // end of barrier
}
\end{lstlisting}

	
	When generating data file we need to name our data file uniquely so that when all of the data files are in one location, no confusion occurs. A good way to do this is to add the time and data stamp at the end of each data file. Although data filename does share a common pattern.
	\input{"Appendix1/codes/time_tracking"}
	%!TEX root = ../thesis.tex
% ******************************* Thesis Appendix A ****************************

The \textbf{src/types.h} file

\begin{lstlisting}[style=CStyle]
using value_type = unsigned long;
using signed_value_type =  long;
\end{lstlisting}
	
	\subsection{Tests}
	A template function is required to run for $l0, l1, l2$. The template argument is the class name. The other two argument is the length and ensemble size we need for one file. The three template arguments are
	\begin{enumerate}
		\item SitePercolation\_ps\_v9
		\item SitePercolationBallisticDeposition\_L1\_v2
		\item SitePercolationBallisticDeposition\_L1\_v2
	\end{enumerate}
	\input{Appendix1/codes/test}
	
	\subsection{Main}
	The main function receives 3 additional command line argument. First one is an integer $l\in\{0,1,2\}$ which determine the range of interaction. Second one is the length of the lattice. And third one is the size of the ensemble. For example, $1 200 5000$ will run the program for $l=1$, $L=200$ for ensemble size of $5000$.
	
	The \textit{run\_in\_main(int, char**)} function is the one where the  \textit{simulate\_site\_percolation\_T<>(size\_t, size\_t)} executes for different classes.
	
	\input{Appendix1/codes/main}
	
	\subsection{CMakeLists}
	All the header and source files are listed here and how the compiler should link them is generated by running cmake. \url{https://cmake.org/cmake-tutorial/}
	
	\input{Appendix1/codes/cmakelists}
	
	\subsection{complete code}
	Complete code for RSBD model on square lattice is available at
	\url{https://github.com/sha314/SqLattice_RSBD}
	or use the git link to clone the repository
	\url{https://github.com/sha314/SqLattice_RSBD.git}
	\\
	Detailed version of the same program with other extensions are available at
	\url{https://github.com/sha314/SqLatticePercolation}
	or the git link
	\url{https://github.com/sha314/SqLatticePercolation.git}



