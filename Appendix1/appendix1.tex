%!TEX root = ../thesis.tex
% ******************************* Thesis Appendix A ****************************
\definecolor{mGreen}{rgb}{0,0.6,0}
\definecolor{mGray}{rgb}{0.5,0.5,0.5}
\definecolor{mPurple}{rgb}{0.58,0,0.82}
\definecolor{backgroundColour}{rgb}{0.95,0.95,0.92}

\lstdefinestyle{CStyle}{
	backgroundcolor=\color{backgroundColour},   
	commentstyle=\color{mGreen},
	keywordstyle=\color{magenta},
	numberstyle=\tiny\color{mGray},
	stringstyle=\color{mPurple},
	basicstyle=\footnotesize,
	breakatwhitespace=false,         
	breaklines=true,                 
	captionpos=b,                    
	keepspaces=true,                 
	numbers=left,                    
	numbersep=5pt,                  
	showspaces=false,                
	showstringspaces=false,
	showtabs=false,                  
	tabsize=2,
	language=C
}


\lstdefinestyle{CMakeStyle}{
	backgroundcolor=\color{backgroundColour},   
	basicstyle=\footnotesize,
	breakatwhitespace=false,         
	breaklines=true,                 
	captionpos=b,                    
	keepspaces=true,                 
	numbers=left,                    
	numbersep=5pt,                  
	showspaces=false,                
	showstringspaces=false,
	showtabs=false,                  
	tabsize=2,
}


\chapter{Percolation} 
\section{Algorithm}
\section{Code}
	
	Each header file starts with a directive \textit{\#ifndef} and \textit{\#define}, which is necessary because one header file is needed multiple times and including it more than once results in error. These directive prevents it. Of course this directive must be closed by \textit{\#endif}.
	
	
	\subsection{Index}
	Here the notion of index of site and index of bond is defined. Site index has two element which determine row and column. Bond Index had three element describing bond type, row, column. Bond type can be horizontal or vertical.
	%!TEX root = ../thesis.tex
% ******************************* Thesis Appendix A ****************************

the \textbf{src/index/index.h} file
\begin{lstlisting}[style=CStyle]
struct Index{
value_type row_{};
value_type column_{};
~Index()                      = default;
Index()                       = default;
Index(value_type x, value_type y) : row_{x}, column_{y} {}
};

class IndexRelative{
public:
int x_{};
int y_{};
~IndexRelative()                      = default;
IndexRelative()                       = default;
IndexRelative(int x, int y) : x_{x}, y_{y} {}
};

struct BondIndex{
BondType bondType;
value_type row_;
value_type column_;
~BondIndex()                        = default;
BondIndex()                         = default;
BondIndex(BondType hv, value_type row, value_type column)
:  row_{row}, column_{column}
{
bondType = hv;
}
bool horizontal() const { return bondType == BondType::Horizontal;}
bool vertical() const { return bondType == BondType::Vertical;}
};

Index get_2nn_in_1nn_direction(Index center, Index nn_1, value_type length);
std::vector<Index> get_2nn_s_in_1nn_s_direction(Index center, const std::vector<Index> &nn_1, value_type length);
\end{lstlisting}


	
	
	
	\subsection{Site}
	The site class contains all information about a site, e.g., if it is active or not and if it is then what is it's group id or relative index.
	%!TEX root = ../thesis.tex
% ******************************* Thesis Appendix A ****************************

The \textbf{src/lattice/site.h} file

\begin{lstlisting}[style=CStyle]
struct Site{
bool _status{false};
int _group_id{-1};
Index _id{};
IndexRelative _relative_index{0,0};
public:
~Site()                 = default;
Site()                  = default;
Site(const Site&)             = default;
Site(Site&&)            = default;
Site& operator=(const Site&)  = default;
Site& operator=(Site&&) = default;
Site(Index id, value_type length){
if(id.row_ >= length || id.column_ >= length){
std::cout << "out of range : line " << __LINE__ << std::endl;
}
_id.row_ = id.row_;
_id.column_ = id.column_;
}

bool isActive() const { return _status;}
void activate(){ _status = true;}
void deactivate() {
_relative_index = {0,0};
_group_id = -1;
_status = false;
}
Index ID() const { return  _id;}

int     get_groupID() const {return _group_id;}
void    set_groupID(int g_id) {_group_id = g_id;}
std::stringstream getSite() const {
std::stringstream ss;
if(isActive())
ss << _id;
else
ss << "(*)";
return ss;
}

void relativeIndex(IndexRelative r){
_relative_index = r;
}
void relativeIndex(int x, int y){
_relative_index = {x,y};
}
IndexRelative relativeIndex() const {return _relative_index;}
};
std::ostream& operator<<(std::ostream& os, const Site& site);
bool operator==(Site& site1, Site& site2);
\end{lstlisting}

	
	\subsection{Bond}
	The bond class contains all information about a bond, e.g., if it is active or not and if it is then what is it's group id.
	%!TEX root = ../thesis.tex
% ******************************* Thesis Appendix A ****************************

The \textbf{src/lattice/bond.h} file

\begin{lstlisting}[style=CStyle]
struct Bond{
bool _status{false};
value_type _length;
int _group_id{-1};
BondType bondType;
IndexRelative _relative_index{0,0};
Index _end1;
Index _end2;
BondIndex _id;

~Bond() = default;
Bond() = default;
Bond(Index end1, Index end2, value_type length){
_end1.row_ = end1.row_;
_end1.column_ = end1.column_;
_end2.row_ = end2.row_;
_end2.column_ = end2.column_;
_length = length;
if(_end1.row_ == _end2.row_){
bondType = BondType::Horizontal;
if(_end1.column_ > _end2.column_){
if(_end1.column_ == _length-1 && _end2.column_ ==0){
}
else{
_end1.column_ = end2.column_;
_end2.column_ = end1.column_;
}
}
else if(_end1.column_ < _end2.column_){
if(_end1.column_ == 0 && _end2.column_ == _length-1){
_end1.column_ = end2.column_;
_end2.column_ = end1.column_;
}
}
}
else if(_end1.column_ == _end2.column_){
bondType = BondType::Vertical;
if(_end1.row_ > _end2.row_){
if(_end1.row_ == _length-1 && _end2.row_ ==0){
}
else{
_end1.row_ = end2.row_;
_end2.row_ = end1.row_;
}
}
else if(_end1.row_ < _end2.row_){
if(_end1.row_ == 0 && _end2.row_ == _length-1){
_end1.row_ = end2.row_;
_end2.row_ = end1.row_;
}
}
}
else{
std::cout << '(' << _end1.row_ << ',' << _end1.column_ << ')' << "<->"
<< '(' << _end2.row_ << ',' << _end2.column_ << ')'
<< " is not a valid bond : line " << __LINE__ << std::endl;
}
_id = BondIndex(bondType, _end1.row_, _end1.column_);  // unsigned long
}

std::vector<Index> getSites() const { return {_end1, _end2};}
Index id() const {return _end1;}
BondIndex ID() const {return _id;}
void activate() {_status = true;}
void deactivate() {
_relative_index = {0,0};
_group_id = -1;
_status = false;
}
bool isActive() const { return _status;}

int get_groupID() const {return _group_id;}
void set_groupID(int g_id) {_group_id = g_id;}

bool isHorizontal() const { return bondType == BondType ::Horizontal;}
bool isVertical()   const { return bondType == BondType ::Vertical;}
void relativeIndex(IndexRelative r){
_relative_index = r;
}
void relativeIndex(int x, int y){
_relative_index = {x,y};
}
IndexRelative relativeIndex() const {return _relative_index;}
};
\end{lstlisting}


The \textbf{src/lattice/bond\_type.cpp} file

\begin{lstlisting}[style=CStyle]
enum class BondType{
Horizontal,
Vertical
};
\end{lstlisting}
			
	\subsection{Lattice}
	The lattice class consists of array of sites and bonds. This class contains information about lattice size. And contains functions to view the lattice differently in the console.
	%!TEX root = ../thesis.tex
% ******************************* Thesis Appendix A ****************************

the \textbf{src/lattice/lattice.h} file
\begin{lstlisting}[style=CStyle]
class SqLattice {
std::vector<std::vector<Site>> _sites;  // holds all the sites
std::vector<std::vector<Bond>> _h_bonds;  // holds all horizontal bonds
std::vector<std::vector<Bond>> _v_bonds;  // holds all vertical bonds
bool _bond_resetting_flag=true; // so that we can reset all bonds
bool _site_resetting_flag=true; // and all sites
value_type _length{};
private:
void reset_sites();
void reset_bonds();
public:
~SqLattice() = default;
SqLattice() = default;
SqLattice(SqLattice&) = default;
SqLattice(SqLattice&&) = default;
SqLattice& operator=(const SqLattice&) = default;
SqLattice& operator=(SqLattice&&) = default;
SqLattice(value_type length, bool activate_bonds, bool activate_sites, bool bond_reset, bool site_reset);
void reset(bool reset_all=false);

void activate_site(Index index);
void activateBond(BondIndex bond);
void deactivate_site(Index index);
void deactivate_bond(Bond bond);
value_type length() const { return  _length;}
Site& getSite(Index index);
Bond& getBond(BondIndex);
const Site& getSite(Index index) const ;
const Bond& getBond(BondIndex index) const ;
void setGroupID(Index index, int group_id);
void setGroupID(BondIndex index, int group_id);
const int getGroupID(Index index)const;
const int getGroupID(BondIndex index)const;

std::vector<Index> get_neighbor_site_indices(Index site);   // site neighbor of site
std::vector<BondIndex> get_neighbor_bond_indices(BondIndex site); // bond neighbor of bond
std::vector<Index> get_neighbor_indices(BondIndex bond);   // two site neighbor of bond.
static std::vector<Index> get_neighbor_site_indices(size_t length, Index site);   // 4 site neighbor of site
static std::vector<BondIndex> get_neighbor_bond_indices(size_t length, BondIndex site); // 6 bond neighbor of bond
static std::vector<Index> get_neighbor_indices(size_t length, BondIndex bond);   // 2 site neighbor of bond.
};
\end{lstlisting}


	
	\subsection{Exception}
	Different types of exceptions.
	%!TEX root = ../thesis.tex
% ******************************* Thesis Appendix A ****************************

The \textbf{src/exception/exception.h} file

\begin{lstlisting}[style=CStyle]
#ifndef PERCOLATION_EXCEPTIONS_H
#define PERCOLATION_EXCEPTIONS_H

#include <string>
#include <iostream>

struct Mismatch{
    std::string msg_;
    size_t line_;
    Mismatch(size_t line, std::string msg="")
            :line_{line}, msg_{msg}  {}

    void what() const {
        std::cerr << msg_ << "\nId and index mismatch at line " << line_ << std::endl;
    }
};

struct InvalidIndex{
    std::string msg_;
    InvalidIndex(std::string msg)  :msg_{msg}  {}

    void what() const {
        std::cerr << msg_ << std::endl;
    }
};

struct InvalidBond{
    std::string msg_;
    InvalidBond(std::string msg)  :msg_{msg}  {}

    void what() const {
        std::cout << msg_ << std::endl;
    }
};

/**
 * When any neighor is occupied and no suitable neighbor is found, throw this exception
 */
struct OccupiedNeighbor{
    std::string msg_;
    OccupiedNeighbor(std::string msg): msg_{msg}{}

    void what() const {
        std::cout << msg_ << std::endl;
    }
};

/**
 * If the 1st or the 2nd nearest neighbor is not valid
 */
struct InvalidNeighbor{
    std::string msg_;
    InvalidNeighbor(std::string msg)  :msg_{msg}  {}

    void what() const {
        std::cout << msg_ << std::endl;
    }
};
#endif //PERCOLATION_EXCEPTIONS_H
\end{lstlisting}



	
	\subsection{Cluster}
	The cluster class contains all information about a cluster. Number of sites and bonds in a cluster and the id of the cluster is contained in a cluster.
	%!TEX root = ../thesis.tex
% ******************************* Thesis Appendix A ****************************

The \textbf{src/percolation/cluster.h} file

\begin{lstlisting}[style=CStyle]
class Cluster{
std::vector<BondIndex>  _bond_index; 
std::vector<Index>      _site_index; 
int _creation_time{-1};  
int _id{-1};
public:

~Cluster()                           = default;
Cluster()                            = default;
Cluster(Cluster&)                 = default;
Cluster(Cluster&&)                = default;
Cluster& operator=(const Cluster&)      = default;
Cluster& operator=(Cluster&&)     = default;

explicit Cluster(int id){
_id = id;       // may be modified in the program

_creation_time = id + 1;       // only readable, not modifiable
}
void addSiteIndex(Index );
void addBondIndex(BondIndex );
Index lastAddedSite(){return _site_index.back();}
BondIndex lastAddedBond(){return _bond_index.back();}
void insert(const std::vector<BondIndex>& bonds);
void insert(const std::vector<Index>& sites);
void insert(const Cluster& cluster);
void insert_v2(const Cluster& cluster);
void insert_with_id_v2(const Cluster& cluster, int id);

friend std::ostream& operator<<(std::ostream& os, const Cluster& cluster);
const std::vector<BondIndex>&   getBondIndices()    {return _bond_index;}
const std::vector<Index>&       getSiteIndices()    {return _site_index;}
const std::vector<BondIndex>&   getBondIndices()  const  {return _bond_index;}
const std::vector<Index>&       getSiteIndices()  const  {return _site_index;}
value_type numberOfBonds() const { return _bond_index.size();}
value_type numberOfSites() const { return _site_index.size();}
int get_ID() const { return _id;}
void set_ID(int id) { _id = id;}
int birthTime() const {return _creation_time;}
Index getRootSite()const{return _site_index[0];} 
BondIndex getRootBond()const{return _bond_index[0];} 
bool empty() const { return _bond_index.empty() && _site_index.empty();}
void clear() {_bond_index.clear(); _site_index.clear(); }
};
\end{lstlisting}


	
	\subsection{Percolation}
	%!TEX root = ../thesis.tex
% ******************************* Thesis Appendix A ****************************

The \textbf{src/percolation/percolation.h} file

\begin{lstlisting}[style=CStyle]
#ifndef SITEPERCOLATION_PERCOLATION_H
#define SITEPERCOLATION_PERCOLATION_H

#include <vector>
#include <set>
#include <unordered_set>
#include <map>
#include <climits>
#include <fstream>


#include "../types.h"
#include "../lattice/lattice.h"
#include "../index/index.h"

#include <random>


/**
* The Square Lattice Percolation class
*/
class SqLatticePercolation{
// constants
value_type  _length;
value_type _max_number_of_bonds;
value_type _max_number_of_sites;
char type{'0'}; // percolation type. 's' -> site percolation. 'b' -> bond percolation
protected:

// structural variables of lattice
SqLattice _lattice;

value_type _index_sequence_position{};

std::vector<Cluster> _clusters;   // check and remove reapeted index manually
// every birthTime we create a cluster we assign an set_ID for them

double _occuption_probability {};
// entropy
double _entropy{};
double _entropy_current{};
size_t _cluster_count{};
value_type _bonds_in_cluster_with_size_two_or_more{0};   // total number of bonds in the clusters. all cluster has bonds > 1
bool _reached_critical = false; // true if the system has reached critical value

value_type _total_relabeling{};
double time_relabel{};
value_type _number_of_occupied_sites{};
value_type _max_iteration_limit{};
std::random_device _random_device;
std::mt19937 _random_generator;

void set_type(char t){type = t;} // setting percolation type
public:
static constexpr const char* signature = "SqLatticePercolation";

virtual ~SqLatticePercolation() = default;
SqLatticePercolation(value_type length);
void reset();


bool occupy();
value_type length() const { return _length;}
value_type maxSites() const {return _max_number_of_sites;}
value_type maxBonds() const { return _max_number_of_bonds;}

/*********
*  I/O functions
*/
virtual void viewCluster();
virtual void viewClusterExtended();
virtual void view_bonds(){
_lattice.view_bonds();
}
virtual void viewLattice(){
_lattice.view_sites();

}

/**
* Also shows the cluster index of the sites
*/
virtual void viewLatticeExtended(){
_lattice.view_sites_extended();
}

/**
* Displays group ids of sites in a matrix form
*/
virtual void viewLatticeByID(){
_lattice.view_sites_by_id();
_lattice.view_bonds_by_id();
}

virtual void viewSiteByID(){
_lattice.view_sites_by_id();
}

virtual void viewBondByID(){
_lattice.view_bonds_by_id();
}

virtual void viewSiteByRelativeIndex(){
_lattice.view_sites_by_relative_index();
}
virtual void viewBondByRelativeIndex(){
_lattice.view_bonds_by_relative_index_v4();
}

virtual void viewByRelativeIndex(){
_lattice.view_by_relative_index();
}

virtual void view(){
_lattice.view();
}

virtual double occupationProbability() const { return _occuption_probability;}
virtual double entropy() { return _entropy_current;}
double entropy_by_site(); // for future convenience. // the shannon entropy. the full calculations. time consuming
double entropy_by_bond(); // for future convenience. // the shannon entropy. the full calculations. time consuming
size_t numberOfcluster() const {return _cluster_count;}

void get_cluster_info(
std::vector<value_type> &site,
std::vector<value_type> &bond
);

char get_type() const {return type;} // get percolation type
virtual value_type maxIterationLimit() {return _max_iteration_limit;};

double get_relabeling_time() const {return time_relabel;}
value_type relabeling_count() const {return _total_relabeling;}
};


/**
* Site Percolation by Placing Sites
*
* version 9
*
* First it randomizes the site index list then use it.
* Paradigm Shift:
* Does not delete cluster only makes it empty so that index and id remains the same.
* This way Searching for index of the cluster using id can be omitted.
*
* Feature :
* 1. Can turn on and off both horizontal and boundary condition
*
* 2. Uses class Cluster_v2 for storing clusters
*
* 3. Uses Group_ID for Bonds and Sites to identify that they are in the same cluster
*
* 4. Occupation probability is calculated by sites,
*      i.e., number of active sites divided by total number of sites
*
* 5. Spanning is calculated by number of bonds in a spanning clusters with periodicity turned off,
*      i.e., number of bonds in the spanning clusters divided by total number of bonds
*
* 6. Unweighted relabeling is ommited in this version ??
*
* 7. Runtime is significantly improved. For example, if L=200 program will take ~1 min to place all sites.
*
* 8. Unnecessary methods of previous version is eliminated
*
* 9. Checking spanning by keeping track of boundary sites is implemented
*
* 10. last modified cluster id can be obtained from @var _last_placed_site
*
*
*/
class SitePercolation_ps_v9 : public SqLatticePercolation{
protected:
// flags to manipulate method
bool _periodicity{false};

value_type min_index; // minimum index = 0
value_type max_index; // maximum index = length - 1

// index sequence
std::vector<Index> index_sequence;  // initialized once
std::vector<value_type> randomized_index;

// every birthTime we create a cluster we assign an set_ID for them
int _cluster_id{};
value_type _index_last_modified_cluster{};  // id of the last modified cluster

// order parameter calculation ingradiants
// id of the cluster which has maximum number of bonds. used to calculate order parameter
value_type _number_of_bonds_in_the_largest_cluster{};
value_type _number_of_sites_in_the_largest_cluster{};   // might be useful later

Index _last_placed_site;    // keeps track of last placed site

/**************
* Spanning variables
************/
/*Holds indices on the edges*/
std::vector<Index> _top_edge, _bottom_edge, _left_edge, _right_edge;

std::vector<Index> _spanning_sites;
std::vector<Index> _wrapping_sites;
std::vector<value_type> number_of_sites_to_span;
std::vector<value_type> number_of_bonds_to_span;

value_type _total_relabeling{};

/*****************************************
* Private Methods
******************************************/
void relabel_sites(const std::vector<Index> &sites, int id_a, int delta_x_ab, int delta_y_ab) ;

double time_relabel{};
public:
static constexpr const char* signature = "SitePercolation_ps_v8";

~SitePercolation_ps_v9() = default;
SitePercolation_ps_v9() = default;
SitePercolation_ps_v9(SitePercolation_ps_v9 & ) = default;
SitePercolation_ps_v9(SitePercolation_ps_v9 && ) = default;
explicit SitePercolation_ps_v9(value_type length, bool periodicity=true);

SitePercolation_ps_v9& operator=(SitePercolation_ps_v9 & ) = default;
//    SitePercolation_ps_v8&& operator=(SitePercolation_ps_v8 && ) = default;
double get_relabeling_time() {return time_relabel;}
value_type relabeling_count() const {return _total_relabeling;}

virtual void reset();


bool periodicity() const {return _periodicity;}
std::string getSignature();


void add_entropy_for_bond(value_type index);
void subtract_entropy_for_bond(const std::set<value_type> &found_index_set, int base=-1);

/*************************************************
* Site placing methods
************************************************/
virtual bool occupy();
value_type placeSite_weighted(Index site); // uses weighted relabeling by first identifying the largest cluster
value_type placeSite_weighted(Index site,
std::vector<Index>& neighbor_sites,
std::vector<BondIndex>& neighbor_bonds);

Index selectSite(); // selecting site

void connection_v2(Index site, std::vector<Index> &site_neighbor, std::vector<BondIndex> &bond_neighbor);

// applicable to weighted relabeling
void relabel_sites_v5(Index root_a, const Cluster& clstr_b); // relative index is set accordingly

/**********************************************
* Information about current state of Class
**********************************************/
double numberOfOccupiedSite() const { return _number_of_occupied_sites;}
double occupationProbability() const { return double(_number_of_occupied_sites)/maxSites();}
double entropy(); // the shannon entropy

value_type numberOfBondsInTheLargestCluster_v2();
value_type numberOfSitesInTheLargestCluster();

value_type numberOfSitesInTheSpanningClusters_v2()  ;
value_type numberOfBondsInTheSpanningClusters_v2()  ;

value_type numberOfSitesInTheWrappingClusters()  ;
value_type numberOfBondsInTheWrappingClusters()  ;

/***********************************
* Spanning Detection
**********************************/
bool detectSpanning_v6(const Index& site);

bool check_if_id_matches(Index site, const std::vector<Index> &edge);

bool detectWrapping();

/************************************
*  Tracker
*  Must be called each time a site is placed
***********************************/
void track_numberOfBondsInLargestCluster();
void track_numberOfSitesInLargestCluster();

/*********************************
* I/O functions
* Printing Status
********************************/
Index lastPlacedSite() const { return _last_placed_site;}

void spanningIndices() const;
void wrappingIndices() const;

/***********************************************
* Visual data for plotting
*********************************************/
// lattice visual data for python
void writeVisualLatticeData(const std::string& filename, bool only_spanning=true);

protected:
void initialize();
void initialize_index_sequence();
void randomize_v2(); // better random number generator

int find_cluster_index_for_placing_new_bonds(const std::vector<Index> &neighbors, std::set<value_type> &found_indices);

value_type manage_clusters(
const std::set<value_type> &found_index_set,
std::vector<BondIndex> &hv_bonds,
Index &site,
int base_id // since id and index is same
);

public:
// on test
IndexRelative getRelativeIndex(Index root, Index site_new);
};

/*******************************************************************************
* Site Percolation Ballistic Deposition
* Extended from SitePercolation_ps_v9
* *************************************************************/
class SitePercolationBallisticDeposition_v2: public SitePercolation_ps_v9{
protected:
// elements of @indices_tmp will be erased if needed but not of @indices
std::vector<value_type> indices;
std::vector<value_type> indices_tmp;
public:
static constexpr const char* signature = "SitePercolation_BallisticDeposition_v2";
virtual ~SitePercolationBallisticDeposition_v2(){
indices.clear();
indices_tmp.clear();
};
SitePercolationBallisticDeposition_v2(value_type length, bool periodicity);

virtual bool occupy();

/************************************
* Site selection methods
*/
Index select_site(std::vector<Index> &sites, std::vector<BondIndex> &bonds);
Index select_site_upto_1nn(std::vector<Index> &sites, std::vector<BondIndex> &bonds);
Index select_site_upto_2nn(std::vector<Index> &sites, std::vector<BondIndex> &bonds);

void reset();
void initialize_indices();

virtual std::string getSignature() {
std::string s = "sq_lattice_site_percolation_ballistic_deposition_";
if(_periodicity)
s += "_periodic_";
else
s += "_non_periodic_";
return s;
}


/***********************************
* occupy upto 1st nearset neighbor.
* If the randomly selected site is occupied then select one of the nearest neighor randomly
* If it is also occupied skip the rest setps and start next iteration Else occupy it
*/
value_type placeSite_1nn_v2();
/*********************************
* occupy upto 2nd nearest neighbor.
* If the randomly selected site is occupied then select one of the nearest neighor randomly
* If it is also occupied, select the next neighbor in the direction of motion Else occupy it.
* If the 2nd nearest neighbor in the direction of motion is also occupied then skip the rest of the steps
*      and start the next iteration
*/
value_type placeSite_2nn_v1();

};

/***********
* Only L1
*/
class SitePercolationBallisticDeposition_L1_v2: public SitePercolationBallisticDeposition_v2{
public:
~SitePercolationBallisticDeposition_L1_v2() = default;
SitePercolationBallisticDeposition_L1_v2(value_type length, bool periodicity)
:SitePercolationBallisticDeposition_v2(length, periodicity){}

bool occupy() {
// if no site is available then return false
if(_number_of_occupied_sites == maxSites()){
return false;
}
try {
value_type v = placeSite_1nn_v2();
_occuption_probability = occupationProbability(); // for super class
return v != ULLONG_MAX;
}catch (OccupiedNeighbor& on){
//        on.what();
return false;
}

}

std::string getSignature() {
std::string s = "sq_lattice_site_percolation_ballistic_deposition_L1";
if(_periodicity)
s += "_periodic_";
else
s += "_non_periodic_";
return s;
}

};

/*********************
*
*/
class SitePercolationBallisticDeposition_L2_v2: public SitePercolationBallisticDeposition_v2{
public:
~SitePercolationBallisticDeposition_L2_v2() = default;
SitePercolationBallisticDeposition_L2_v2(value_type length, bool periodicity)
:SitePercolationBallisticDeposition_v2(length, periodicity){}

bool occupy() {
// if no site is available then return false

if(_number_of_occupied_sites == maxSites()){
return false;
}

try {

//            value_type v = placeSite_2nn_v0();
value_type v = placeSite_2nn_v1();
_occuption_probability = occupationProbability(); // for super class

return v != ULLONG_MAX;
}catch (OccupiedNeighbor& on){
//        on.what();
//        cout << "line : " << __LINE__ << endl;
return false;
}

}

std::string getSignature() {
std::string s = "sq_lattice_site_percolation_ballistic_deposition_L2";
if(_periodicity)
s += "_periodic_";
else
s += "_non_periodic_";
return s;
}

};

#endif //SITEPERCOLATION_PERCOLATION_H

\end{lstlisting}

The \textbf{src/percolation/percolation.cpp} file

\begin{lstlisting}[style=CStyle]
#include "percolation.h"

using namespace std;

/**
*
* @param length
*/
SqLatticePercolation::SqLatticePercolation(value_type length) {
if (length <= 2) {
/*
* Because if _length=2
* there are total of 4 distinct bond. But it should have been 8, i.e, (2 * _length * _length = 8)
*/
cerr << "_length <= 2 does not satisfy _lattice properties for percolation : line" << __LINE__ << endl;
exit(1);
}
_length = length;
value_type _length_squared = length * length;
_max_number_of_bonds = 2*_length_squared;
_max_number_of_sites = _length_squared;
_clusters = vector<Cluster>();

//    size_t seed = 0;
//    cerr << "automatic seeding is commented : line " << __LINE__ << endl;
auto seed = _random_device();
_random_generator.seed(seed); // seeding
cout << "seeding with " << seed << endl;
}


/**
*
*/
void SqLatticePercolation::viewCluster() {
cout << "clusters with numberOfBonds greater than 1" << endl;
value_type total_bonds{}, total_sites{};

for (value_type i{}; i != _clusters.size(); ++i) {
if(_clusters[i].empty()){
//            cout << "Empty cluster : line " << endl;
continue;
}
cout << "cluster [" << i << "] : " << '{' << endl;
cout << _clusters[i];
total_bonds += _clusters[i].numberOfBonds();
total_sites += _clusters[i].numberOfSites();

cout << '}' << endl;
}
cout << "Total bonds " << total_bonds << endl;
cout << "Total sites " << total_sites << endl;
}



/**
* Extended version of view_cluster
*/
void SqLatticePercolation::viewClusterExtended() {
cout << "clusters with numberOfBonds greater than 1" << endl;
value_type total_bonds{}, total_sites{};

std::vector<Index> sites;
std::vector<BondIndex> bonds;
for (value_type i{}; i != _clusters.size(); ++i) {
if(_clusters[i].empty()){
//            cout << "Empty cluster : line " << endl;
continue;
}
cout << "cluster [" << i << "] : ID (" << _clusters[i].get_ID() << "){" << endl;
// printing sites
sites = _clusters[i].getSiteIndices();
cout << "Sites : size (" << sites.size() << ") : ";
cout << '{';
for (auto a: sites) {
cout << a << ',';
}
cout << '}' << endl;

bonds = _clusters[i].getBondIndices();
cout << "Bonds : size (" << bonds.size() << ") : ";
cout << '{';
for (auto a: bonds) {
if (a.horizontal()) {
// horizontal bond
cout << _lattice.getBond({BondType::Horizontal, a.row_, a.column_}) << ',';
} else if (a.vertical()) {
// vertical bond
cout << _lattice.getBond({BondType::Vertical, a.row_, a.column_}) << ',';
} else {
cout << '!' << a << '!' << ','; // bond is not valid
}
}
cout << '}';

cout << endl;

total_bonds += _clusters[i].numberOfBonds();
total_sites += _clusters[i].numberOfSites();

cout << '}' << endl;
}
cout << "Total bonds " << total_bonds << endl;
cout << "Total sites " << total_sites << endl;
}

/**
*
* @param site
* @param bond
* @param total_site
* @param total_bond
*/
void
SqLatticePercolation::get_cluster_info(
vector<value_type> &site,
vector<value_type> &bond
) {
value_type total_site{}, total_bond{};
site.clear();
bond.clear();

unsigned long size = _clusters.size();
site.reserve(size);
bond.reserve(size);

value_type s, b;

for(value_type i{}; i < size; ++i){
if(_clusters[i].empty()){
//            cout << "Empty cluster : line " << endl;
continue;
}
s = _clusters[i].numberOfSites();
b = _clusters[i].numberOfBonds();
site.push_back(s);
bond.push_back(b);
total_site += s;
total_bond += b;
}
if(site.size() != bond.size()){
cout << "Size mismatched : line " << __LINE__ << endl;
}
//    cout << "total bonds " << total_bond << endl;
//    cout << "tatal sites " << total_site << endl;
if(type == 's'){
for(value_type j{total_bond}; j < maxBonds(); ++j){
bond.push_back(1); // cluster of length 1
total_bond += 1;
}
}
if(type == 'b'){
for(value_type j{total_site}; j < maxSites(); ++j){
total_site += 1;
site.push_back(1); // cluster of length 1
}
}
}

void SqLatticePercolation::reset() {
_lattice.reset();
_clusters.clear();
_index_sequence_position = 0;

_occuption_probability = 0;
// entropy
_entropy=0;
_entropy_current=0;
_total_relabeling = 0;
time_relabel = 0;
_cluster_count = 0;
_reached_critical = false;
}


/**
* Entropy calculation is performed here. The fastest method possible.
* Cluster size is measured by site.
* @return current entropy of the lattice
*/
double SqLatticePercolation::entropy_by_site() {
double H{}, mu ;

for(size_t i{}; i < _clusters.size(); ++i){
if(!_clusters[i].empty()){
mu = _clusters[i].numberOfSites() / double(_number_of_occupied_sites);
H += mu*log(mu);
}
}

return -H;
}

/**
* Entropy calculation is performed here. The fastest method possible.
* Cluster size is measured by site.
* @return current entropy of the lattice
*/
double SqLatticePercolation::entropy_by_bond() {
double H{}, mu ;

for(size_t i{}; i < _clusters.size(); ++i){
if(!_clusters[i].empty()){
mu = _clusters[i].numberOfBonds() / double(maxBonds());
H += mu*log(mu);
}
}

double number_of_cluster_with_size_one = maxBonds() - _bonds_in_cluster_with_size_two_or_more;
//    cout << " _bonds_in_cluster_with_size_two_or_more " << _bonds_in_cluster_with_size_two_or_more << " : line " << __LINE__ << endl;
mu = 1.0/double(maxBonds());
H += number_of_cluster_with_size_one * log(mu) * mu;

return -H;
}
\end{lstlisting}
	%!TEX root = ../thesis.tex
% ******************************* Thesis Appendix A ****************************

The \textbf{src/percolation/percolation\_site\_v9.cpp} file

\begin{lstlisting}[style=CStyle]
#include <cstdlib>
#include <climits>
#include <unordered_set>
#include <mutex>

#include "percolation.h"

#include "../util/printer.h"
#include <omp.h>
#include <thread>
#include <algorithm>

#include "../util/time_tracking.h"

using namespace std;



/**
*
* @param length       : length of the lattice
* @param impure_sites : number of impure sites. cannot be greater than length*length
*/
SitePercolation_ps_v9::SitePercolation_ps_v9(value_type length, bool periodicity)
:SqLatticePercolation(length)
{
std::cout << "Constructing SitePercolation_ps_v9 object : line " << __LINE__ << endl;
SqLatticePercolation::set_type('s');

_periodicity = periodicity;
_index_sequence_position = 0;
_lattice = SqLattice(length, true, false, false, true);   // since it is a site percolation all bonds will be activated by default

min_index = 0;
max_index = length - 1;

index_sequence.resize(maxSites());
randomized_index.resize(maxSites());
_max_iteration_limit = maxSites();

initialize_index_sequence();
initialize();
randomize_v2();  // randomize the untouched_site_indices
}


/**
*
*/
void SitePercolation_ps_v9::initialize() {

// to improve performence
number_of_sites_to_span.reserve(maxSites());
number_of_bonds_to_span.reserve(maxSites());

_top_edge.reserve(length());
_bottom_edge.reserve(length());
_left_edge.reserve(length());
_right_edge.reserve(length());

//    randomized_index_sequence = index_sequence;
}


/**
* Called only once when the object is constructed for the first time
*/
void SitePercolation_ps_v9::initialize_index_sequence() {
value_type m{}, n{};
for (value_type i{}; i != index_sequence.size(); ++i) {
randomized_index[i] = i;
index_sequence[i] = Index(m, n);
++n;
if (n == length()) {
n = 0;
++m;
}
}
//for (value_type i{}; i != index_sequence.size(); ++i) {cout << index_sequence[i] << endl;}
}


/**
* Reset all calculated values and then call initiate()
* to initiallize for reuse
*
* caution -> it does not erase _calculation_flags, for it will be used for calculation purposes
*/
void SitePercolation_ps_v9::reset() {
SqLatticePercolation::reset();
// variables
_number_of_occupied_sites = 0;
_index_sequence_position = 0;
_cluster_id = 0;

// containers
number_of_sites_to_span.clear();
number_of_bonds_to_span.clear();
_spanning_sites.clear();
_wrapping_sites.clear();
_bonds_in_cluster_with_size_two_or_more = 0;
_index_last_modified_cluster = 0;  // id of the last modified cluster
_number_of_bonds_in_the_largest_cluster = 0;
_number_of_sites_in_the_largest_cluster = 0;
// clearing edges
_top_edge.clear();
_bottom_edge.clear();
_left_edge.clear();
_right_edge.clear();
initialize();
randomize_v2();
time_relabel = 0;
_total_relabeling = 0;
}


/**
* Randomize the indices
*/
void SitePercolation_ps_v9::randomize_v2(){

std::shuffle(randomized_index.begin(), randomized_index.end(), _random_generator);
//    cout << "Index sequence : " << randomized_index_sequence << endl;
}


/*************************************************
* Calculation methods
*
***********************************/

/*
* Instead of calculating entropy for 1000s of cluster in every iteration
* just keep track of entropy change, i.e.,
* how much to subtract and how much to add.
*/
/**
* Must be called before merging the clusters
* @param found_index_set
*/
void SitePercolation_ps_v9::subtract_entropy_for_bond(const set<value_type> &found_index, int base){
double nob, mu_bond, H{};
if(base >= 0){
nob = _clusters[base].numberOfBonds();
mu_bond = nob / maxBonds();
H += log(mu_bond) * mu_bond;
}
for(auto x : found_index){
nob = _clusters[x].numberOfBonds();
mu_bond = nob / maxBonds();
H += log(mu_bond) * mu_bond;
}
_entropy -= -H;
}



/**
* Must be called after merging the clusters
* Cluster length is measured by bonds
* @param index
*/
void SitePercolation_ps_v9::add_entropy_for_bond(value_type index){
double nob = _clusters[index].numberOfBonds();
double mu_bond = nob / maxBonds();
double H = log(mu_bond) * mu_bond;
_entropy += -H;
}



/**
* Condition: must be called each time a site is placed
*/
void SitePercolation_ps_v9::track_numberOfBondsInLargestCluster() {

// calculating number of bonds in the largest cluster // by cluster index
// checking number of bonds
if(_clusters[_index_last_modified_cluster].numberOfBonds() > _number_of_bonds_in_the_largest_cluster){
_number_of_bonds_in_the_largest_cluster = _clusters[_index_last_modified_cluster].numberOfBonds();
}

}

/**
*
*/
void SitePercolation_ps_v9::track_numberOfSitesInLargestCluster(){

// calculating number of bonds in the largest cluster // by cluster index
// checking number of bonds
if(_clusters[_index_last_modified_cluster].numberOfSites() > _number_of_sites_in_the_largest_cluster){
_number_of_sites_in_the_largest_cluster = _clusters[_index_last_modified_cluster].numberOfSites();
}
}


/**
*
* @param neighbors         :
* @param found_index_set   : index of the clusters that will be merged together.
*                            Does not contain the base cluster index or id.
* @return                  : id of the base cluster
*/
int
SitePercolation_ps_v9::find_cluster_index_for_placing_new_bonds(
const vector<Index> &neighbors, std::set<value_type> &found_index_set
){
found_index_set.clear();
value_type size{}, tmp{}, index, base{ULONG_MAX};
int base_id{-1};
int id;
for (auto n: neighbors) {
id = _lattice.getGroupID(n);
if(id >=0) {
index = value_type(id);
tmp = _clusters[index].numberOfSites();
if(tmp > size){
size = tmp;
base_id = id;
base = index;
}

found_index_set.insert(index);

}
}
found_index_set.erase(base);
return base_id;
}


/**
* Last placed site is added to a cluster. If this connects other clusters then merge all
* cluster together to get one big cluster. All sites that are part of the other clusters
* are relabled according to the id of the base cluster.
* @param found_index_set : index of the clusters that are neighbors of the last placed site
* @param hv_bonds        : bonds that connects the last placed site and its neighbors
*                          and which are not part of any cluster of size larger than one
* @param site            : last placed site
* @param base_id         : id of the base cluster
* @return
*/
value_type SitePercolation_ps_v9::manage_clusters(
const set<value_type> &found_index_set,
vector<BondIndex> &hv_bonds,
Index &site,
int base_id
)
{


if (base_id != -1) {
value_type base = value_type(base_id); // converting here
_clusters[base].addSiteIndex(site);
int id_base = _clusters[base].get_ID();
vector<Index> neibhgors = _lattice.get_neighbor_site_indices(site);
// find which of the neighbors are of id_base as the base cluster
IndexRelative r;
for(auto n: neibhgors){
if(_lattice.getGroupID(n) == id_base){
// find relative index with respect to this site
r = getRelativeIndex(n, site);
break; // since first time r is set running loop is doing no good
}
}

// put_values_to_the_cluster new values in the 0-th found index
_clusters[base].insert(hv_bonds);
_lattice.getSite(site).relativeIndex(r);
_lattice.setGroupID(site, id_base); // relabeling for 1 site

// merge clusters with common values from all other cluster        // merge clusters with common values from all other cluster


for(value_type ers: found_index_set){

_total_relabeling += _clusters[ers].numberOfSites(); // only for debugging purposes
// perform relabeling on the sites
relabel_sites_v5(site, _clusters[ers]);

// store values of other found indices to the cluster
_clusters[base].insert_v2(_clusters[ers]);
_cluster_count--; // reducing number of clusters
_clusters[ers].clear(); // emptying the cluster

}
_index_last_modified_cluster = base;


} else {
// create new element for the cluster
_clusters.push_back(Cluster(_cluster_id));
value_type _this_cluster_index = _clusters.size() -1;
_lattice.setGroupID(site, _cluster_id); // relabeling for 1 site
_cluster_count++; // increasing number of clusters
_cluster_id++;
_clusters.back().insert(hv_bonds);
_clusters[_this_cluster_index].addSiteIndex(site);
_index_last_modified_cluster = _this_cluster_index;   // last cluster is the place where new bonds are placed

}
return _index_last_modified_cluster;
}





/**
* Relative index of site_new with respect to root
* @param root
* @param site_new
* @return
*/
IndexRelative SitePercolation_ps_v9::getRelativeIndex(Index root, Index site_new){
//    cout << "Entry \"SitePercolation_ps_v9::getRelativeIndex\" : line " << __LINE__ << endl;
int delta_x = -int(root.column_) + int(site_new.column_); // if +1 then root is on the right ??
int delta_y = int(root.row_) - int(site_new.row_); // if +1 then root is on the top ??


// normalizing delta_x
if(delta_x > 1){
delta_x /= -delta_x;
}
else if(delta_x < -1){
delta_x /= delta_x;
}

// normalizing delta_y
if(delta_y > 1){
delta_y /= -delta_y;
}else if(delta_y < -1){
delta_y /= delta_y;
}

IndexRelative indexRelative_root = _lattice.getSite(root).relativeIndex();
//    cout << "Relative index of root " << indexRelative_root << endl;
//    cout << "Delta x,y " << delta_x << ", " << delta_y << endl;
IndexRelative r =  {indexRelative_root.x_ + delta_x, indexRelative_root.y_ + delta_y};
//    cout << "Relative index of site_new " << r << endl;
return r;
}



/**
* Take a bond index only if the corresponding site is active
* takes longer? time than version 1?, i.e.,  connection()
* @param site
* @param site_neighbor
* @param bond_neighbor
*/
void SitePercolation_ps_v9::connection_v2(Index site, vector<Index> &site_neighbor, vector<BondIndex> &bond_neighbor)
{

value_type prev_column  = (site.column_ + length() - 1) % length();
value_type prev_row     = (site.row_ + length() - 1) % length();
value_type next_row     = (site.row_ + 1) % length();
value_type next_column  = (site.column_ + 1) % length();

if(!_periodicity){
// without periodicity
if (site.row_ == min_index) { // top edge including corners
if(site.column_ == min_index){
// upper left corner

site_neighbor.resize(2);
site_neighbor[0] = {site.row_, next_column};
site_neighbor[1] = {next_row, site.column_};

bond_neighbor.reserve(2);
if(!_lattice.getSite(site_neighbor[0]).isActive()){
bond_neighbor.push_back({BondType::Horizontal, site.row_, site.column_});
}
if(!_lattice.getSite(site_neighbor[1]).isActive()){
bond_neighbor.push_back({BondType::Vertical, site.row_, site.column_});
}

return;

}
else if(site.column_ == max_index){
// upper right corner

site_neighbor.resize(2);
site_neighbor[0] = {site.row_, prev_column};
site_neighbor[1] = {next_row, site.column_};

bond_neighbor.reserve(2);
if(!_lattice.getSite(site_neighbor[0]).isActive()){
bond_neighbor.push_back({BondType::Horizontal, site.row_, prev_column});
}
if(!_lattice.getSite(site_neighbor[1]).isActive()){
bond_neighbor.push_back({BondType::Vertical, site.row_, site.column_});
}

return;
}
else{
// top edge excluding corners
site_neighbor.resize(3);
site_neighbor[0] = {site.row_, next_column};
site_neighbor[1] = {site.row_, prev_column};
site_neighbor[2] = {next_row, site.column_};

bond_neighbor.reserve(4);
if(!_lattice.getSite(site_neighbor[0]).isActive()) {
bond_neighbor.push_back({BondType::Horizontal, site.row_, site.column_});
}
if(!_lattice.getSite(site_neighbor[1]).isActive()){
bond_neighbor.push_back({BondType::Horizontal, site.row_, prev_column});
}
if(!_lattice.getSite(site_neighbor[2]).isActive()){
bond_neighbor.push_back({BondType::Vertical,    site.row_, site.column_});
}

return;

}
}
else if (site.row_ == max_index) { // bottom edge including corners
if (site.column_ == min_index) {
// lower left corner
site_neighbor.resize(2);
site_neighbor[0] = {site.row_, next_column};
site_neighbor[1] = {prev_row, site.column_};

bond_neighbor.reserve(2);
if(!_lattice.getSite(site_neighbor[0]).isActive()){
bond_neighbor.push_back({BondType::Horizontal, site.row_, site.column_});
}
if(!_lattice.getSite(site_neighbor[1]).isActive()){
bond_neighbor.push_back({BondType::Vertical, prev_row, site.column_});
}


return;

} else if (site.column_ == max_index) {
// lower right corner
site_neighbor.resize(2);
site_neighbor[0] = {site.row_, prev_column};
site_neighbor[1] = {prev_row, site.column_};

bond_neighbor.reserve(2);
if(!_lattice.getSite(site_neighbor[0]).isActive()){
bond_neighbor.push_back({BondType::Horizontal, site.row_, prev_column});
}
if(!_lattice.getSite(site_neighbor[1]).isActive()){
bond_neighbor.push_back({BondType::Vertical, prev_row, site.column_});
}

return;

} else {
// bottom edge excluding corners
//  bottom edge
site_neighbor.resize(3);
site_neighbor[0] = {site.row_, next_column};
site_neighbor[1] = {site.row_, prev_column};
site_neighbor[2] = {prev_row, site.column_};

bond_neighbor.reserve(3);
if(!_lattice.getSite(site_neighbor[0]).isActive()) {
bond_neighbor.push_back({BondType::Horizontal, site.row_, site.column_});
}
if(!_lattice.getSite(site_neighbor[1]).isActive()){
bond_neighbor.push_back({BondType::Horizontal, site.row_, prev_column});
}
if(!_lattice.getSite(site_neighbor[2]).isActive()){
bond_neighbor.push_back({BondType::Vertical, prev_row, site.column_});
}

return;
}
}
/* site.x_ > min_index && site.x_ < max_index &&  is not possible anymore*/
else if (site.column_ == min_index) { // left edge not in the corners
site_neighbor.resize(3);
site_neighbor[0] = {site.row_, next_column};
site_neighbor[1] = {next_row, site.column_};
site_neighbor[2] = {prev_row, site.column_};

bond_neighbor.reserve(3);
if(!_lattice.getSite(site_neighbor[0]).isActive()) {
bond_neighbor.push_back({BondType::Horizontal, site.row_, site.column_});
}
if(!_lattice.getSite(site_neighbor[1]).isActive()){
bond_neighbor.push_back({BondType::Vertical,    site.row_, site.column_});
}
if(!_lattice.getSite(site_neighbor[2]).isActive()){
bond_neighbor.push_back({BondType::Vertical, prev_row, site.column_});
}

return;
}
else if (site.column_ == max_index) {
// right edge no corners

site_neighbor.resize(3);
site_neighbor[0] = {site.row_, prev_column};
site_neighbor[1] = {next_row, site.column_};
site_neighbor[2] = {prev_row, site.column_};

bond_neighbor.reserve(3);
if(!_lattice.getSite(site_neighbor[0]).isActive()){
bond_neighbor.push_back({BondType::Horizontal, site.row_, prev_column});
}
if(!_lattice.getSite(site_neighbor[1]).isActive()){
bond_neighbor.push_back({BondType::Vertical,    site.row_, site.column_});
}
if(!_lattice.getSite(site_neighbor[2]).isActive()){
bond_neighbor.push_back({BondType::Vertical, prev_row, site.column_});
}

return;
}

}
// 1 level inside the lattice
// not in any the boundary
site_neighbor.resize(4);
site_neighbor[0] = {site.row_, next_column};
site_neighbor[1] = {site.row_, prev_column};
site_neighbor[2] = {next_row, site.column_};
site_neighbor[3] = {prev_row, site.column_};

bond_neighbor.reserve(4);
if(!_lattice.getSite(site_neighbor[0]).isActive()) {
bond_neighbor.push_back({BondType::Horizontal, site.row_, site.column_});
}
if(!_lattice.getSite(site_neighbor[1]).isActive()){
bond_neighbor.push_back({BondType::Horizontal, site.row_, prev_column});
}
if(!_lattice.getSite(site_neighbor[2]).isActive()){
bond_neighbor.push_back({BondType::Vertical,    site.row_, site.column_});
}
if(!_lattice.getSite(site_neighbor[3]).isActive()) {
bond_neighbor.push_back({BondType::Vertical, prev_row, site.column_});
}

}



/**
*
* @param site
* @param edge
* @return
*/
bool SitePercolation_ps_v9::check_if_id_matches(Index site, const vector<Index> &edge){
for(auto s :edge){
if(_lattice.getGroupID(site) == _lattice.getGroupID(s)){
// no need to put the site here
return true;
}
}
return false;
}



/***********************************************
*  Placing sites
*
*****************************************/

/**
* All site placing method in one place
*
* @return true if operation is successfull
*/
bool SitePercolation_ps_v9::occupy() {
if(_index_sequence_position >= maxSites()){
return false;
}
Index site = selectSite();
placeSite_weighted(site);
_occuption_probability = occupationProbability(); // for super class
return true;
}

/***
* Index of the selected site must be provided with the argument
*
* Wrapping and spanning index arrangement is enabled.
* Entropy is calculated smoothly.
* Entropy is measured by site and bond both.
* @param current_site
* @return
*/
value_type SitePercolation_ps_v9::placeSite_weighted(Index current_site) {
// randomly choose a site
if (_number_of_occupied_sites == maxSites()) {
return ULONG_MAX;// unsigned long int maximum value
}

_last_placed_site = current_site;
_lattice.activate_site(current_site);
++_number_of_occupied_sites;
// find the bonds for this site
vector<BondIndex> bonds;
vector<Index>     sites;
connection_v2(current_site, sites, bonds);
_bonds_in_cluster_with_size_two_or_more += bonds.size();

// find one of hv_bonds in _clusters and add ever other value to that place. then erase other position
set<value_type> found_index_set;
int  base_id = find_cluster_index_for_placing_new_bonds(sites, found_index_set);

subtract_entropy_for_bond(found_index_set, base_id);  // tracking entropy change
value_type merged_cluster_index = manage_clusters(
found_index_set, bonds, current_site, base_id
);
add_entropy_for_bond(merged_cluster_index); // tracking entropy change
// running tracker
track_numberOfBondsInLargestCluster(); // tracking number of bonds in the largest cluster
track_numberOfSitesInLargestCluster();
return merged_cluster_index;
}

/***
* Index of the selected site must be provided with the argument
*
* Wrapping and spanning index arrangement is enabled.
* Entropy is calculated smoothly.
* Entropy is measured by site and bond both.
* @param current_site
* @return
*/
value_type SitePercolation_ps_v9::placeSite_weighted(
Index current_site,
vector<Index>& neighbor_sites,
vector<BondIndex>& neighbor_bonds
) {
// randomly choose a site
if (_number_of_occupied_sites == maxSites()) {
return ULONG_MAX;// unsigned long int maximum value
}
_bonds_in_cluster_with_size_two_or_more += neighbor_bonds.size();
_last_placed_site = current_site;
_lattice.activate_site(current_site);
++_number_of_occupied_sites;
// find one of hv_bonds in _clusters and add ever other value to that place. then erase other position
set<value_type> found_index_set;
int  base_id = find_cluster_index_for_placing_new_bonds(neighbor_sites, found_index_set);
subtract_entropy_for_bond(found_index_set, base_id);  // tracking entropy change
value_type merged_cluster_index = manage_clusters(
found_index_set, neighbor_bonds, current_site, base_id
);
add_entropy_for_bond(merged_cluster_index); // tracking entropy change
// running tracker
track_numberOfBondsInLargestCluster(); // tracking number of bonds in the largest cluster
track_numberOfSitesInLargestCluster();
return merged_cluster_index;
}



/**
*
* @return
*/
Index SitePercolation_ps_v9::selectSite(){
//    Index current_site = randomized_index_sequence[_index_sequence_position]; // old
value_type index = randomized_index[_index_sequence_position];
Index current_site = index_sequence[index]; // new process
++_index_sequence_position;
return current_site;
}


/***************************************************
* View methods
****************************************/


/**
*
*/
void SitePercolation_ps_v9::spanningIndices() const {
cout << "Spanning Index : id" << endl;
for(Index i: _spanning_sites){
cout << "Index " << i << " : id " << _lattice.getGroupID(i) << endl;
}
}

void SitePercolation_ps_v9::wrappingIndices() const {
cout << "Wrapping Index : id : relative index" << endl;
for(auto i: _wrapping_sites){
cout << "Index " << i << " : id "
<< _lattice.getGroupID(i)
<< " relative index : " << _lattice.getSite(i).relativeIndex() << endl;
}
}


/****************************************
* Spanning Detection
****************************************/


/**
* success : gives correct result
* length       time
* 200          7.859000 sec
* 500          2 min 18.874000 sec
* only check for the cluster id of the recently placed site
* @param site : Check spanning for this argument
* @return
*/
bool SitePercolation_ps_v9::detectSpanning_v6(const Index& site) {
//    cout << "Entry -> detectSpanning_v4() : line " << __LINE__ << endl;
if(_periodicity) {
cout << "Cannot detect spanning if _periodicity if ON: line " << __LINE__ << endl;
return false;
}
if(_reached_critical ){
return true;  // we have already reached critical point
}

// first check if the site with a cluster id is already a spanning site
for(const Index& ss: _spanning_sites){
if(_lattice.getSite(ss).get_groupID() == _lattice.getSite(site).get_groupID()){
//            cout << "Already a spanning site : line " << __LINE__ << endl;
return true;
}
}

// only check for the newest site placed
if(site.row_ == min_index){ // top index
if(!check_if_id_matches(site, _top_edge)) {
_top_edge.push_back(site);
}
}
else if(site.row_ == max_index){
if(!check_if_id_matches(site, _bottom_edge)){
_bottom_edge.push_back(site);
}
}

// checking column indices for Left-Right boundary
if(site.column_ == min_index){ // left edge
if(!check_if_id_matches(site, _left_edge)) {
_left_edge.push_back(site);
}
}
else if(site.column_ == max_index){
if(!check_if_id_matches(site, _right_edge)) {
_right_edge.push_back(site);
}
}

if(_number_of_occupied_sites < length()){
//        cout << "Not enough site to span : line " << __LINE__ << endl;
return false;
}


vector<Index>::iterator it_top = _top_edge.begin();
vector<Index>::iterator it_bot = _bottom_edge.begin();
bool found_spanning_site = false;
int id = _lattice.getGroupID(site);

if(_top_edge.size() < _bottom_edge.size()){
// if matched found on the smaller edge look for match in the larger edge
for(; it_top < _top_edge.end(); ++it_top){
if(id == _lattice.getGroupID(*it_top)){
for(; it_bot < _bottom_edge.end(); ++it_bot){
if(id == _lattice.getGroupID(*it_bot)){
// match found !
if(!check_if_id_matches(*it_top ,_spanning_sites)) {
_reached_critical = true;
_spanning_sites.push_back(*it_top);
}
found_spanning_site = true;
_bottom_edge.erase(it_bot);
}
}

if(found_spanning_site){
found_spanning_site = false;
_top_edge.erase(it_top);
}

}
}
}else{
for (; it_bot < _bottom_edge.end(); ++it_bot) {
if (id == _lattice.getGroupID(*it_bot)) {
for (; it_top < _top_edge.end(); ++it_top) {
if (id == _lattice.getGroupID(*it_top)) {
// match found !
if (!check_if_id_matches(*it_top, _spanning_sites)) {
_reached_critical = true;
_spanning_sites.push_back(*it_top);
}
found_spanning_site = true;
_top_edge.erase(it_top);
}
}
if(found_spanning_site){
found_spanning_site = false;
_bottom_edge.erase(it_top);
}
}
}

}

found_spanning_site = false;
vector<Index>::iterator it_lft = _left_edge.begin();
vector<Index>::iterator it_rht = _right_edge.begin();

if(_left_edge.size() < _right_edge.size()){
for(; it_lft < _left_edge.end(); ++it_lft) {
if (id == _lattice.getGroupID(*it_lft)) {
for (; it_rht < _right_edge.end(); ++it_rht) {
if (id == _lattice.getGroupID(*it_rht)) {
if (!check_if_id_matches(*it_lft, _spanning_sites)) {
_spanning_sites.push_back(*it_lft);
_reached_critical = true;
}
found_spanning_site = true;
_right_edge.erase(it_rht);
}
}
if (found_spanning_site) {
found_spanning_site = false;
_left_edge.erase(it_lft);
}
}
}
}else{
for (; it_rht < _right_edge.end(); ++it_rht) {
if (id == _lattice.getGroupID(*it_rht)) {
for(; it_lft < _left_edge.end(); ++it_lft) {
if (id == _lattice.getGroupID(*it_lft)) {
if (!check_if_id_matches(*it_lft, _spanning_sites)) {
_spanning_sites.push_back(*it_lft);
_reached_critical = true;
}
found_spanning_site = true;
_left_edge.erase(it_lft);
}
}
if (found_spanning_site) {
found_spanning_site = false;
_right_edge.erase(it_rht);
}
}
}
}


// now do the matching with left and right for horizontal spanning
// meaning new site is added to _spanning_sites so remove them from top and bottom edges



// filter spanning ids


return !_spanning_sites.empty();

}



/***********************************
* Wrapping Detection
**********************************/
/**
* Wrapping is detected here using the last placed site
* @return bool. True if wrapping occured.
*/
bool SitePercolation_ps_v9::detectWrapping() {
Index site = lastPlacedSite();
// only possible if the cluster containing 'site' has sites >= length of the lattice
if(_number_of_occupied_sites < length()){
return false;
}

if(_reached_critical){
return true; // reached critical in previous step
}
// check if it is already a wrapping site
int id = _lattice.getGroupID(site);
int tmp_id{};
for (auto i: _wrapping_sites){
tmp_id = _lattice.getGroupID(i);
if(id == tmp_id ){
return true;
}
}

// get four neighbors of site always. since wrapping is valid if periodicity is implied
vector<Index> sites = _lattice.get_neighbor_site_indices(site);

if(sites.size() < 2){ // at least two neighbor of  site is required
return false;
}else{
IndexRelative irel = _lattice.getSite(site).relativeIndex();
//        cout << "pivot's " << site << " relative " << irel << endl;
IndexRelative b;
for (auto a:sites){
if(_lattice.getGroupID(a) != _lattice.getGroupID(site)){
// different cluster
continue;
}
// belongs to the same cluster
b = _lattice.getSite(a).relativeIndex();
//            cout << "neibhbor " << a << " relative " << b << endl;
if(abs(irel.x_ - b.x_) > 1 || abs(irel.y_ - b.y_) > 1){
//                cout << "Wrapping : line " << __LINE__ << endl;
_wrapping_sites.push_back(site);
_reached_critical = true;
return true;
}
}
}
// if %_wrapping_indices is not empty but wrapping is not detected for the current site (%site)
// that means there is wrapping but not for the %site
return !_wrapping_sites.empty();
}

/********************************************************************
* Relabeling
*
*********************************************************************/
/**
* Relabels site and also reassign relative index to the relabeled sites
*
* @param site_a  : last added site index of the base cluster
* @param clstr_b : 2nd cluster, which to be merged withe the root
*/
void SitePercolation_ps_v9::relabel_sites_v5(Index site_a, const Cluster& clstr_b) {
const vector<Index> sites = clstr_b.getSiteIndices();
int id_a = _lattice.getGroupID(site_a);
int id_b = clstr_b.get_ID();
Index b = clstr_b.getRootSite();

// get four site_b of site_a
vector<Index> sites_neighbor_a = _lattice.get_neighbor_site_indices(site_a);
Index site_b;
IndexRelative relative_index_b_after;
bool flag{false};
// find which site_b has id_a of clstr_b
for(auto n: sites_neighbor_a){
if(id_b == _lattice.getGroupID(n)){
// checking id_a equality is enough. since id_a is the id_a of the active site already.
relative_index_b_after = getRelativeIndex(site_a, n);
site_b = n;
flag = true;
break;
}
}
if(!flag){
cout << "No neibhgor found! : line " << __LINE__ << endl;
}

IndexRelative relative_site_a = _lattice.getSite(site_a).relativeIndex();
// with this delta_a and delta_y find the relative index of site_b while relative index of site_a is known
IndexRelative relative_site_b_before = _lattice.getSite(site_b).relativeIndex();
int delta_x_ab = relative_index_b_after.x_ - relative_site_b_before.x_;
int delta_y_ab = relative_index_b_after.y_ - relative_site_b_before.y_;
relabel_sites(sites, id_a, delta_x_ab, delta_y_ab);
}



void SitePercolation_ps_v9::relabel_sites(const vector<Index> &sites, int id_a, int delta_x_ab, int delta_y_ab)  {
int x, y;
Index a;
IndexRelative relative_site__a;
for (value_type i = 0; i < sites.size(); ++i) {
a = sites[i];
_lattice.setGroupID(a, id_a);
relative_site__a = _lattice.getSite(a).relativeIndex();
x = relative_site__a.x_ + delta_x_ab;
y = relative_site__a.y_ + delta_y_ab;
_lattice.getSite(a).relativeIndex(x, y);
}
}



/**********************************************
* Information about current state of Class
**********************************************/

/**
* Entropy calculation is performed here. The fastest method possible.
* Cluster size is measured by bond.
* @return current entropy of the lattice
*/
double SitePercolation_ps_v9::entropy() {
double H{};
double number_of_cluster_with_size_one = maxBonds() - _bonds_in_cluster_with_size_two_or_more;
//    cout << " _bonds_in_cluster_with_size_two_or_more " << _bonds_in_cluster_with_size_two_or_more << " : line " << __LINE__ << endl;
double mu = 1.0/double(maxBonds());
H += number_of_cluster_with_size_one * log(mu) * mu;
H *= -1;
_entropy_current =  _entropy + H;
return _entropy_current;
}



/**
* Only applicable if the number of bonds in the largest cluster is calculated when occupying the lattice.
* Significantly efficient than the previous version numberOfBondsInTheLargestCluster()
* @return
*/
value_type SitePercolation_ps_v9::numberOfBondsInTheLargestCluster_v2() {
//    return _clusters[_index_largest_cluster].numberOfBonds();
return _number_of_bonds_in_the_largest_cluster;
}



/**
*
* @return
*/
value_type SitePercolation_ps_v9::numberOfSitesInTheLargestCluster() {
value_type  len{}, nob{};
for(auto c: _clusters){
nob = c.numberOfSites();
if (len < nob){
len = nob;
}
}
_number_of_sites_in_the_largest_cluster = len;
return len;
}


/**********************************
* Spanning methods
**********************************/

/**
*
* @return
*/
value_type SitePercolation_ps_v9::numberOfSitesInTheSpanningClusters_v2() {

if(! _spanning_sites.empty()){
int id = _lattice.getGroupID(_spanning_sites.front());
if(id >= 0) {
return _clusters[id].numberOfSites();
}
}
return 0;
}


/**
*
* @return
*/
value_type SitePercolation_ps_v9::numberOfBondsInTheSpanningClusters_v2() {
if(!_spanning_sites.empty()){
//        cout << "number of spanning sites " << _spanning_sites.size() << " : line " << __LINE__ << endl;
int id = _lattice.getGroupID(_spanning_sites.front());
if(id >= 0) {
return _clusters[id].numberOfBonds();
}
}
return 0;
}

/**
*
* @return
*/
value_type SitePercolation_ps_v9::numberOfSitesInTheWrappingClusters(){
value_type nos{};
int id{};
for(auto i: _wrapping_sites){
id = _lattice.getGroupID(i);
if(id >= 0) {
nos += _clusters[id].numberOfSites();
}
}
return nos;
}

/**
*
* @return
*/
value_type SitePercolation_ps_v9::numberOfBondsInTheWrappingClusters(){
value_type nob{};
int id{};
for(auto i: _wrapping_sites){
id = _lattice.getGroupID(i);
if(id >= 0) {
nob += _clusters[id].numberOfBonds();
}
}
return nob;
}


std::string SitePercolation_ps_v9::getSignature() {
string s = "sq_lattice_site_percolation";
if(_periodicity)
s += "_periodic_";
else
s += "_non_periodic_";
return s;
}

/**
*
* @param filename
* @param only_spanning
*/
void SitePercolation_ps_v9::writeVisualLatticeData(const string &filename, bool only_spanning) {
std::ofstream fout(filename);
ostringstream header_info;
header_info << "{"
<< "\"length\":" << length()
<< ",\"signature\":\"" << getSignature() << "\""
<< ",\"x\":\"" << lastPlacedSite().column_ << "\""
<< ",\"y\":\"" << lastPlacedSite().row_ << "\""
<< "}" ;

fout << "#" << header_info.str() << endl;
fout << "#<x>,<y>,<color>" << endl;
fout << "# color=0 -means-> unoccupied site" << endl;
int id{-1};
if(!_spanning_sites.empty()){
id = _lattice.getGroupID(_spanning_sites.front());
}
else if(!_wrapping_sites.empty()){
id = _lattice.getGroupID(_wrapping_sites.front());
}

if(only_spanning){
if(id < 0){
cerr << "id < 0 : line " << __LINE__ << endl;
}
vector<Index> sites = _clusters[id].getSiteIndices();
for(auto s: sites){
fout << s.column_ << ',' << s.row_ << ',' << id << endl;
}
}
else {
for (value_type y{}; y != length(); ++y) {
for (value_type x{}; x != length(); ++x) {
id = _lattice.getGroupID({y, x});
if(id != -1) {
fout << x << ',' << y << ',' << id << endl;
}
}
}
}
fout.close();
}

\end{lstlisting}

The \textbf{src/percolation/percolation\_site\_ballistic\_deps\_v2.cpp} file

\begin{lstlisting}[style=CStyle]
#include <cstdlib>
#include <climits>

#include "percolation.h"

using namespace std;

/**
*
* @param length
*/
SitePercolationBallisticDeposition_v2::SitePercolationBallisticDeposition_v2(value_type length, bool periodicity)
: SitePercolation_ps_v9(length, periodicity )
{

std::cout << "Constructing SitePercolationBallisticDeposition_v2 object : line " << __LINE__ << endl;

initialize_indices();
indices_tmp = indices;
//    randomize_index();
}

/**
*
*/
void SitePercolationBallisticDeposition_v2::reset() {
SitePercolation_ps_v9::reset();
indices_tmp = indices;
}

/**
* Called only once when the object is constructed for the first time
*/
void SitePercolationBallisticDeposition_v2::initialize_indices() {
indices = vector<value_type>(maxSites());
for(value_type i{}; i != indices.size(); ++i){
indices[i] = i; // assign index first
}
}



/*******************************************
* Site selection methods
*/

/**
*
* @param sites
* @param bonds
* @return
*/
Index SitePercolationBallisticDeposition_v2::select_site(vector<Index> &sites, vector<BondIndex> &bonds) {
// randomly choose a site
value_type r = std::rand() % (indices_tmp.size());

Index current_site = index_sequence[indices_tmp[r]];
cout << "current site " << current_site << endl;
// find the bonds for this site

if (_lattice.getSite(current_site).isActive()){
indices_tmp.erase(indices_tmp.begin()+r);

throw OccupiedNeighbor{"all of the 1nd neighbors are occupied : line " + std::to_string(__LINE__)};
}

cout << "choosing " << current_site << " out of the neighbors : line " << __LINE__ << endl;
sites.clear();
bonds.clear();
connection_v2(current_site, sites, bonds);
return current_site;
}

/**
*
* @param sites
* @param bonds
* @return
*/
Index SitePercolationBallisticDeposition_v2::select_site_upto_1nn(
vector<Index> &sites, vector<BondIndex> &bonds
) {
// randomly choose a site
value_type r = _random_generator() % (indices_tmp.size());

Index current_site = index_sequence[indices_tmp[r]];
//    cout << "current site " << current_site << endl;
// find the bonds for this site

//    connection_v1(current_site, sites, bonds);
connection_v2(current_site, sites, bonds);

if (_lattice.getSite(current_site).isActive()){ // if the current site is occupied or active
value_type r2 = _random_generator() % (sites.size());
current_site = sites[r2]; // select one of the neighbor randomly

if(_lattice.getSite(current_site).isActive()){
// if the neighbor is also occupied cancel current step
bool flag = true;
//            cout << "if one of the neighbor is inactive. it's engouh to go on" << endl;
for(auto s : sites){
//                cout << s << "->";
if(!_lattice.getSite(s).isActive()){
// if one of the neighber is unoccupied then
flag = false;
//                    cout << " inactive" << endl;
break;
}
//                cout << " active"<< endl;
}
if(flag){
// erase the index, since its four neighbors are occupied
indices_tmp.erase(indices_tmp.begin()+r);
throw OccupiedNeighbor{"all of the 1nd neighbors are occupied : line " + std::to_string(__LINE__)};
}
throw OccupiedNeighbor{"selected 1st neighbor is occupied : line " + std::to_string(__LINE__)};
}
}

//    cout << "choosing " << current_site << " out of the neighbors : line " << __LINE__ << endl;
sites.clear();
bonds.clear();
connection_v2(current_site, sites, bonds);
return current_site;
}



/**
* Select neighbor upto 2nd nearest neighbor
* uses direcion of motion when selecting 2nd nearest neighbor
* @param r : index of sites in the randomized array
* @param sites
* @param bonds
* @return
*/
Index SitePercolationBallisticDeposition_v2::select_site_upto_2nn(
vector<Index> &sites, vector<BondIndex> &bonds
){
value_type r = _random_generator() % (indices_tmp.size());

Index central_site = index_sequence[indices_tmp[r]];
Index selected_site;
// find the bonds for this site


connection_v2(central_site, sites, bonds);

if (_lattice.getSite(central_site).isActive()){
bool flag_nn1 = true; // true means all 1st nearest neighbors are occupied
bool flag_nn2 = true; // true means all 2nd nearest neighbors are occupied
//        cout << "if one of the neighbor is inactive. it's engouh to go on" << endl;
for(auto s : sites){
//            cout << s << "->";
if(!_lattice.getSite(s).isActive()){
// if one of the neighber is unoccupied then
flag_nn1 = false;
//                cout << " inactive" << endl;
break;
}
//            cout << " active"<< endl;
}

value_type r2 = _random_generator() % (sites.size());
Index nn1 = sites[r2]; // select one of the neighbor randomly
//        cout << "nn1 " << nn1 << " : line " << __LINE__ <<endl;
Index nn2;
if(_lattice.getSite(nn1).isActive()){
// if the neighbor is also occupied then choose the 2nd nearest neighbor in the direction of motion
nn2 = get_2nn_in_1nn_direction(central_site, nn1, length());
if(!_periodicity){
// if periodic boundary condition is not enabled then sites on the opposite edges will not contribute
vector<Index> tmp_sites;
vector<BondIndex> tmp_bonds;
// will find all possible neighbors of the selected first nearest neighbor
connection_v2(nn1, tmp_sites, tmp_bonds);
bool valid{false};
for(auto s: tmp_sites){
if(nn2 == s){
//                        cout << "valid 2nd nearest neighbor : line " << __LINE__ << endl;
valid = true;
break;
}
}
if(!valid){
throw InvalidNeighbor{"invalid 2nd nearest neighbor : line " + std::to_string(__LINE__)};
}
}
//            cout << "nn2 " << nn2 << " : line " << __LINE__ <<endl;
// if it is also occupied the skip the step
if(_lattice.getSite(nn2).isActive()) {
flag_nn2 = true;

vector<Index> nn2_sites = get_2nn_s_in_1nn_s_direction(central_site, sites, length());
for(auto x: nn2_sites){
if(!_lattice.getSite(x).isActive()){
flag_nn2 = false;
//                        cout << "inactive";
break;
}
}

if(flag_nn1 && flag_nn2){
// erase the index, since its 1st nearest neighbors are occupied
// and 2nd nearest neighbors are also occupied
indices_tmp.erase(indices_tmp.begin()+r);
}

throw OccupiedNeighbor{"2nd neighbor is also occupied : line " + std::to_string(__LINE__)};
}else{
selected_site = nn2;
}
}else {
selected_site = nn1;
}

sites.clear();
bonds.clear();

connection_v2(selected_site, sites, bonds);
}else{
selected_site = central_site;
}
return selected_site;
}



/********************************************************
* SitePercolationBallisticDeposition_v2
* select upto 1st nearest neighbor
*/

/**
*
* @return
*/
bool SitePercolationBallisticDeposition_v2::occupy() {
// if no site is available then return false

if(_number_of_occupied_sites == maxSites()){
return false;
}

try {

value_type v = placeSite_1nn_v2();
_occuption_probability = occupationProbability(); // for super class


return v != ULLONG_MAX;
}catch (OccupiedNeighbor& on){
//        on.what();
//        cout << "line : " << __LINE__ << endl;
return false;
}

}


/**
*
* 1. Randomly select a site from all sites
* 2. If it is not occupied occupy it.
* 3. If it is occupied select one of the 4 neighbor to occupy
* 4. If the selected neighbor is also occupied cancel current step
* 4. form cluster and track all informations
* 5. go to step 1
* 6. untill spanning cluster appears or no unoccupied site
*/
value_type SitePercolationBallisticDeposition_v2::placeSite_1nn_v2() {

vector<BondIndex> bonds;
vector<Index>     sites;

_last_placed_site = select_site_upto_1nn(sites, bonds);

return placeSite_weighted(_last_placed_site, sites, bonds);
}

/**
*
* @return
*/
value_type SitePercolationBallisticDeposition_v2::placeSite_2nn_v1() {
vector<BondIndex> bonds;
vector<Index>     sites;

try {
_last_placed_site = select_site_upto_2nn(sites, bonds);
return placeSite_weighted(_last_placed_site, sites, bonds);
//    return placeSite_v11(_last_placed_site);
}catch (OccupiedNeighbor& e){
//        cout << "Exception !!!!!!!!!!!!!!!!!!" << endl;
//        e.what();
}catch (InvalidNeighbor& b){
//        cout << "Exception !!!!!!!!!!!!!!!!!!" << endl;
//        b.what();
}
return ULONG_MAX;
}
\end{lstlisting}
	
	\subsection{Utilities}
	%!TEX root = ../thesis.tex
% ******************************* Thesis Appendix A ****************************

The \textbf{src/util/printer.h} file

\begin{lstlisting}[style=CStyle]

template <typename T>
std::ostream& operator<<(std::ostream& os, const std::vector<T> & vec){
os << '{';
for(auto a: vec){
os << a << ',';
}
return os << '}';
}

template <typename T>
std::ostream& operator<<(std::ostream& os, const std::set<T> & vec){
os << '{';
for(auto a: vec){
os << a << ',';
}
return os << '}';
}

template <typename T>
std::ostream& operator<<(std::ostream& os, const std::unordered_set<T> & vec){
os << '{';
for(auto a: vec){
os << a << ',';
}
return os << '}';
}

template <typename K, typename V>
std::ostream& operator<<(std::ostream& os, const std::map<K, V> & m){
os << '{';
for(auto a: m){
os << '(' << a.first << "->" << a.second << "),";
}
return os << '}';
};

template <typename K, typename V>
std::ostream& operator<<(std::ostream& os, const std::unordered_map<K, V> & m){
os << '{';
for(auto a: m){
os << '(' << a.first << "->" << a.second << "),";
}
return os << '}';
};

void print_h_barrier(size_t n, const std::string& initial, const std::string& middles, const std::string& end="\n");

#endif //PERCOLATION_PRINTER_H
\end{lstlisting}

The \textbf{src/util/printer.cpp} file

\begin{lstlisting}[style=CStyle]
void print_h_barrier(size_t n, const string& initial, const string& middles, const string& end){
	cout << initial;
	for(size_t i{}; i < n ; ++i){
		cout << middles;
	}
	cout << end; // end of barrier
}
\end{lstlisting}

	\input{"Appendix1/codes/time_tracking"}
	%!TEX root = ../thesis.tex
% ******************************* Thesis Appendix A ****************************

The \textbf{src/types.h} file

\begin{lstlisting}[style=CStyle]
#ifndef SITEPERCOLATION_TYPES_H
#define SITEPERCOLATION_TYPES_H

using value_type = unsigned long;
using signed_value_type =  long;

#endif //SITEPERCOLATION_TYPES_H
\end{lstlisting}
	
	\subsection{Test}
	%!TEX root = ../thesis.tex
% ******************************* Thesis Appendix A ****************************

The \textbf{src/test/test\_percolation.h} file

\begin{lstlisting}[style=CStyle]
/**
* @tparam PType : Template type of percolation class
* @param argc   : argc from commandline
* @param argv   : argv from commandline
*/
template<class PType>
void simulate_site_percolation_T(value_type length, value_type ensemble_size) {

std::cout << "length " << length << " ensemble_size " << ensemble_size << std::endl;

value_type length_squared = length*length;
value_type twice_length_squared = 2 * length_squared;

PType lattice_percolation(length, true);

std::ostringstream header_info;
header_info << "{"
<< "\"length\":" << length
<< ",\"ensemble_size\":" << ensemble_size
<< ",\"signature\":\"" << lattice_percolation.getSignature() << "\""
<< "}" ;

std::string tm = getCurrentTime();

std::string filename_s = lattice_percolation.getSignature() + "_cluster_by_site_" + std::to_string(length) + '_' + tm;
std::string filename_b = lattice_percolation.getSignature() + "_cluster_by_bond_" + std::to_string(length) + '_' + tm;
std::string filename_critical = lattice_percolation.getSignature() + "_critical_" + std::to_string(length) + '_' + tm;
std::string filename_entropy_order_parameter = lattice_percolation.getSignature()  + std::to_string(length) + '_' + tm;

filename_s += ".csv";
filename_b += ".csv";
filename_critical += ".csv";
filename_entropy_order_parameter += ".csv";


std::ofstream fout_s(filename_s);
// JSON formated header
fout_s << '#' << header_info.str() << std::endl;
fout_s << "#each line is an independent realization" << std::endl;
fout_s << "#each line contains information about all clusters at critical point" << std::endl;
fout_s << "#cluster size is measured by number of sites in it" << std::endl;

std::ofstream fout_b(filename_b);
// JSON formated header
fout_b << '#' << header_info.str() << std::endl;
fout_b << "#each line is an independent realization" << std::endl;
fout_b << "#each line contains information about all clusters at critical point" << std::endl;
fout_b << "#cluster size is measured by number of bonds in it" << std::endl;

std::ofstream fout_critical(filename_critical);
fout_critical << '#' << header_info.str() << std::endl;
fout_critical << "#data at critical occupation probability or pc" << std::endl;
fout_critical << "#<pc>,<sites in wrapping cluster>,<bonds in wrapping cluster>" << std::endl;

// simulation starts here
value_type counter{};
std::vector<double> entropy(lattice_percolation.maxIterationLimit());
std::vector<double> nob_wraping(lattice_percolation.maxIterationLimit()),
nob_largest(lattice_percolation.maxIterationLimit());

for(value_type i{} ; i != ensemble_size ; ++i){

lattice_percolation.reset();

bool successful = false;
auto t_start = std::chrono::system_clock::now();
counter = 0;
bool wrapping_written{false};
while (true){
successful = lattice_percolation.occupy();
if(successful) {
entropy[counter] += lattice_percolation.entropy();
nob_wraping[counter] += lattice_percolation.numberOfBondsInTheWrappingClusters();
nob_largest[counter] += lattice_percolation.numberOfBondsInTheLargestCluster_v2();

if(!wrapping_written && lattice_percolation.detectWrapping()){
fout_critical << lattice_percolation.occupationProbability() << ","
<< lattice_percolation.numberOfSitesInTheWrappingClusters() << ","
<< lattice_percolation.numberOfBondsInTheWrappingClusters()  << std::endl;

std::vector<value_type> site, bond;

lattice_percolation.get_cluster_info(site, bond);

for(value_type j{}; j != site.size(); ++j){
fout_s << site[j] << ',';
}
for(value_type j{}; j != bond.size(); ++j){
fout_b << bond[j] <<',';
}


fout_s << std::endl;
fout_b << std::endl;
wrapping_written = true;
}


++counter;
}
if(counter >= lattice_percolation.maxIterationLimit()){ // twice_length_squared is the number of bonds
break;
}
}

auto t_end = std::chrono::system_clock::now();
std::cout << "Iteration " << i   << " . Elapsed time " << std::chrono::duration<double>(t_end - t_start).count() << " sec" << std::endl;

}
fout_b.close();
fout_s.close();
fout_critical.close();


std::ofstream fout(filename_entropy_order_parameter);
fout << '#' << header_info.str() << std::endl;
fout << "#<p>,<H(p,L)>,<P1(p,L)>,<P2(p,L)>" << std::endl;
fout << "#p = occupation probability" << std::endl;
fout << "#H(p,L) = Entropy = sum( - u_i * log(u_i))" << std::endl;
fout << "#P1(p,L) = Order parameter = (number of bonds in largest cluster) / (total number of bonds)" << std::endl;
fout << "#P2(p,L) = Order parameter = (number of bonds in spanning or wrapping cluster) / (total number of bonds)" << std::endl;
fout << "#C(p,L) = Specific heat = -T dH/dT" << std::endl;
fout << "#X(p,L) = Susceptibility = dP/dp" << std::endl;
fout << "#u_i = (number of bonds in the i-th cluster) / (total number of bonds)" << std::endl;
for(size_t i{}; i < lattice_percolation.maxIterationLimit(); ++i){
fout << (i+1) / double(lattice_percolation.maxIterationLimit()) << ",";
fout << entropy[i] / double(ensemble_size) << ",";
fout << nob_largest[i] / double(ensemble_size /* lattice_percolation.maxBonds()*/) << ",";
fout << nob_wraping[i] / double(ensemble_size /* lattice_percolation.maxBonds()*/) ;
fout << std::endl;
}
fout.close();
}
\end{lstlisting}
	
	\subsection{Main}
	The main function receives 3 additional command line argument. First one is an integer $l\in{0,1,2}$ which determine the range of interaction. Second one is the length of the lattice. And third one is the size of the ensemble. For example, $1 200 5000$ will run the program for $l=1$, $L=200$ for ensemble size of $5000$
	%!TEX root = ../thesis.tex
% ******************************* Thesis Appendix A ****************************

The \textbf{src/main.cpp} file

\begin{lstlisting}[style=CStyle]
#include <iostream>
#include <fstream>
#include <ctime>
#include <chrono>
#include <thread>
#include <mutex>

#include "lattice/lattice.h"
#include "percolation/percolation.h"
#include "util/time_tracking.h"
#include "util/printer.h"

#include "tests/test_percolation.h"


using namespace std;


/****
*  All the function that is run in main
* @param argc
* @param argv
*/
void run_in_main(int argc, char** argv){

int l = atoi(argv[1]);
value_type length = atoi(argv[2]);
value_type ensemble_size = atoi(argv[3]);

if(l==1) {
cout << "Simulating site percolation for l=1" << endl;
simulate_site_percolation_T<SitePercolationBallisticDeposition_L1_v2>(length, ensemble_size); // 2018.11.03
}
else if(l==2) {
cout << "Simulating site percolation for l=2" << endl;
simulate_site_percolation_T<SitePercolationBallisticDeposition_L2_v2>(length, ensemble_size); // 2018.11.03
}else{
cout << "Simulating site percolation for l=0" << endl;
simulate_site_percolation_T<SitePercolation_ps_v9>(length, ensemble_size);
}
}




/**************************************
*  The main function
*
***************************************/
int main(int argc, char** argv) {

cout << "Running started at : " << currentTime() << endl;
cout << "Compiled on        : " << __DATE__ << "\t at " << __TIME__ << endl;
std::cout << "Percolation in a Square Lattice" << std::endl;
auto t_start = std::chrono::system_clock::now();

time_t seed = time(NULL);
srand(seed);    // seeding

run_in_main(argc, argv);

auto t_end= std::chrono::system_clock::now();
std::chrono::duration<double> drtion = t_end - t_start;
std::time_t end_time = std::chrono::system_clock::to_time_t(t_end);
cout << "Program finished at " << std::ctime(&end_time) << endl;
std::cout << "Time elapsed "   << getFormattedTime(drtion.count()) << std::endl;
return 0;
}
\end{lstlisting}
	
	\subsection{CMakeLists}
	%!TEX root = ../thesis.tex
% ******************************* Thesis Appendix A ****************************

The \textbf{CMakeLists.txt} file

\begin{lstlisting}[style=CMakeStyle]
cmake_minimum_required(VERSION 3.0)
project(SqLatticePercolation)

set(CMAKE_CXX_STANDARD 11)

#set (CMAKE_C_COMPILER               /usr/bin/gcc)
#set (CMAKE_CXX_COMPILER             /home/shahnoor/software/pgi/linux86-64-llvm/2018/bin/pgc++)
#set (CMAKE_MAKE_PROGRAM             /usr/bin/make)
SET( CMAKE_CXX_FLAGS  "-pthread -fopenmp")
#SET( CMAKE_EXE_LINKER_FLAGS  "${CMAKE_EXE_LINKER_FLAGS} ${GCC_COVERAGE_LINK_FLAGS}" )

set(SOURCE_FILES
src/main.cpp
src/types.h
src/exception/exceptions.h
src/index/index.cpp
src/index/index.h
src/lattice/bond.cpp
src/lattice/bond.h
src/lattice/bond_type.h
src/lattice/lattice.cpp
src/lattice/lattice.h
src/lattice/site.cpp
src/lattice/site.h
src/percolation/cluster.cpp
src/percolation/cluster.h
src/percolation/percolation.cpp
src/percolation/percolation.h
src/percolation/percolation_site_ballistic_deps_v2.cpp
src/util/printer.h
src/util/time_tracking.cpp
src/util/time_tracking.h
src/util/printer.cpp
src/percolation/percolation_site_v9.cpp
src/tests/test_percolation.h)

add_executable(SqLatticePercolation ${SOURCE_FILES})
\end{lstlisting}
	
	\subsection{complete code}
	Complete code for RSBD model on square lattice is available at
	\url{https://github.com/sha314/SqLattice_RSBD}
	or use the git link to clone the repository
	\url{https://github.com/sha314/SqLattice_RSBD.git}
	\\
	Detailed version of the same program with other extensions are available at
	\url{https://github.com/sha314/SqLatticePercolation}
	or the git link
	\url{https://github.com/sha314/SqLatticePercolation.git}



