%!TEX root = ../thesis.tex
% ******************************* Thesis Appendix A ****************************

The \textbf{src/lattice/site.h} file

\begin{lstlisting}[style=CStyle]
#ifndef SITEPERCOLATION_SITE_H
#define SITEPERCOLATION_SITE_H

#include <array>
#include <set>
#include <vector>
#include <iostream>
#include <memory>

#include "../index/index.h"
#include "../types.h"


/**
* single Site of a lattice
*/
struct Site{
/**
* if true -> site is placed.
* if false -> the (empty) position is there but the site is not (required for site percolation)
*/
bool _status{false};
int _group_id{-1};
Index _id{};

//relative distance from the root site. {0,0} if it is the root site
//for detecting wrapping
IndexRelative _relative_index{0,0};


public:

~Site()                 = default;
Site()                  = default;
Site(const Site&)             = default;
Site(Site&&)            = default;
Site& operator=(const Site&)  = default;
Site& operator=(Site&&) = default;

Site(Index id, value_type length){
// I have handle _neighbor or corner points and edge points carefully
if(id.row_ >= length || id.column_ >= length){
std::cout << "out of range : line " << __LINE__ << std::endl;
}
_id.row_ = id.row_;
_id.column_ = id.column_;
}


bool isActive() const { return _status;}
void activate(){ _status = true;}
void deactivate() {
_relative_index = {0,0};
_group_id = -1;
_status = false;
}
Index ID() const { return  _id;}
/*
* Group get_ID is the set_ID of the cluster they are in
*/
int     get_groupID() const {return _group_id;}
void    set_groupID(int g_id) {_group_id = g_id;}

std::stringstream getSite() const {
std::stringstream ss;
if(isActive())
ss << _id;
else
ss << "(*)";
return ss;
}


void relativeIndex(IndexRelative r){
_relative_index = r;
}

void relativeIndex(int x, int y){
_relative_index = {x,y};
}

IndexRelative relativeIndex() const {return _relative_index;}
};

std::ostream& operator<<(std::ostream& os, const Site& site);
bool operator==(Site& site1, Site& site2);
#endif
\end{lstlisting}

The \textbf{src/lattice/site.cpp} file

\begin{lstlisting}[style=CStyle]
#include <iomanip>
#include "site.h"

std::ostream& operator<<(std::ostream& os, const Site& site)
{
if(site.isActive())
return os << site._id;
else
return os << "(*)";
}


bool operator==(Site& site1, Site& site2){
return (site1.ID().row_ == site2.ID().row_) && (site1.ID().column_ == site2.ID().column_);
}
\end{lstlisting}