%!TEX root = ../thesis.tex
% ******************************* Thesis Appendix A ****************************

The \textbf{src/percolation/cluster.h} file

\begin{lstlisting}[style=CStyle]
#ifndef SITEPERCOLATION_CLUSTER_H
#define SITEPERCOLATION_CLUSTER_H

#include <vector>
#include <set>
#include "../lattice/bond.h"
#include "../types.h"
#include "../lattice/site.h"


/**
* Cluster of bonds and sites
* version 3
* final goal -> make a template cluster. so that we can use it for Bond cluster or Site cluster
* root site (bond) is the first site (bond) of the cluster. nedeed for (wrapping) site percolation
*/
class Cluster{
// contains bond and site
std::vector<BondIndex>  _bond_index; // BondIndex for indexing bonds
std::vector<Index>      _site_index; // Site index

int _creation_time{-1};  // holds the creation birthTime of a cluster object
int _id{-1};
public:
//    using iterator = std::vector<Bond>::iterator;

~Cluster()                           = default;
Cluster()                            = default;
Cluster(Cluster&)                 = default;
Cluster(Cluster&&)                = default;
Cluster& operator=(const Cluster&)      = default;
Cluster& operator=(Cluster&&)     = default;

explicit Cluster(int id){

_id = id;       // may be modified in the program

/*
* Only readable, not modifiable.
* when time = 0 => only lattice exists and bonds in site percolation, not any sites
* When id = 0, time = 1 => we have placed the first site, hence created a cluster with size greater than 1
*      Only then Cluster constructor is called.
*
*/
_creation_time = id + 1;       // only readable, not modifiable
}


void addSiteIndex(Index );
void addBondIndex(BondIndex );

Index lastAddedSite(){return _site_index.back();}
BondIndex lastAddedBond(){return _bond_index.back();}


void insert(const std::vector<BondIndex>& bonds);
void insert(const std::vector<Index>& sites);

void insert(const Cluster& cluster);
void insert_v2(const Cluster& cluster);
void insert_with_id_v2(const Cluster& cluster, int id);


friend std::ostream& operator<<(std::ostream& os, const Cluster& cluster);

const std::vector<BondIndex>&   getBondIndices()    {return _bond_index;}
const std::vector<Index>&       getSiteIndices()    {return _site_index;}

const std::vector<BondIndex>&   getBondIndices()  const  {return _bond_index;}
const std::vector<Index>&       getSiteIndices()  const  {return _site_index;}

value_type numberOfBonds() const { return _bond_index.size();}
value_type numberOfSites() const { return _site_index.size();}
int get_ID() const { return _id;}
void set_ID(int id) { _id = id;}

int birthTime() const {return _creation_time;}

Index getRootSite()const{return _site_index[0];} // for site percolation
BondIndex getRootBond()const{return _bond_index[0];} // for bond percolation
bool empty() const { return _bond_index.empty() && _site_index.empty();}
void clear() {_bond_index.clear(); _site_index.clear(); }
};

#endif //SITEPERCOLATION_CLUSTER_H
\end{lstlisting}

The \textbf{src/percolation/cluster.cpp} file

\begin{lstlisting}[style=CStyle]

#include "cluster.h"

using namespace std;

// add Site index
void Cluster::addSiteIndex(Index index) {
_site_index.push_back(index);
}

void Cluster::addBondIndex(BondIndex bondIndex) {
_bond_index.push_back(bondIndex);
}



void Cluster::insert(const std::vector<BondIndex>& bonds){
_bond_index.reserve(bonds.size());
for(value_type i{} ; i != bonds.size() ; ++i){
_bond_index.push_back(bonds[i]);
}
}

void Cluster::insert(const std::vector<Index>& sites){
_site_index.reserve(sites.size());
for(value_type i{} ; i != sites.size() ; ++i){
_site_index.push_back(sites[i]);
}
}

/**
* Merge two cluster as one
* All intrinsic property should be considered, e.g., creation time of a cluster must be recalculated
* @param cluster
*/
void Cluster::insert(const Cluster &cluster) {
if(_id > cluster._id){
cout << "_id > cluster._id : line " << __LINE__ << endl;
_id = cluster._id;
}
// older time or smaller time is the creation birthTime of the cluster
//    cout << "Comparing " << _creation_time << " and " << cluster._creation_time;
_creation_time = _creation_time < cluster._creation_time ? _creation_time : cluster._creation_time;
//    cout << " Keeping " << _creation_time << endl;
_bond_index.insert(_bond_index.end(), cluster._bond_index.begin(), cluster._bond_index.end());
_site_index.insert(_site_index.end(), cluster._site_index.begin(), cluster._site_index.end());
}


/**
* Merge two cluster as one
* All intrinsic property should be considered, e.g., creation time of a cluster must be recalculated
* @param cluster
*/
void Cluster::insert_v2(const Cluster &cluster) {
// older time or smaller time is the creation birthTime of the cluster
//    cout << "Comparing " << _creation_time << " and " << cluster._creation_time;
_creation_time = _creation_time < cluster._creation_time ? _creation_time : cluster._creation_time;
//    cout << " Keeping " << _creation_time << endl;
_bond_index.insert(_bond_index.end(), cluster._bond_index.begin(), cluster._bond_index.end());
_site_index.insert(_site_index.end(), cluster._site_index.begin(), cluster._site_index.end());
}


void Cluster::insert_with_id_v2(const Cluster &cluster, int id) {
_id = id;
// older time or smaller time is the creation birthTime of the cluster
//    cout << "Comparing " << _creation_time << " and " << cluster._creation_time;
_creation_time = _creation_time < cluster._creation_time ? _creation_time : cluster._creation_time;
//    cout << " Keeping " << _creation_time << endl;
_bond_index.insert(_bond_index.end(), cluster._bond_index.begin(), cluster._bond_index.end());
_site_index.insert(_site_index.end(), cluster._site_index.begin(), cluster._site_index.end());
}


std::ostream &operator<<(std::ostream &os, const Cluster &cluster) {
os << "Sites : size (" << cluster._site_index.size() << ") : ";
os << '{';
for(auto a: cluster._site_index){
os << a << ',';
}
os << '}' << endl;

os << "Bonds : size (" << cluster._bond_index.size() <<") : ";
os << '{';
for(auto a: cluster._bond_index){
os << a << ',';
}
os << '}';

return os << endl;
}
\end{lstlisting}